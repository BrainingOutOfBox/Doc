\subsection{Schlussfolgerungen}
Im Kapitel der Schlussfolgerungen wollen wir nochmals auf unser Projekt und dessen Verlauf schauen und unsere Ergebnisse kritisch bewerten. Dabei wollen wir aufzeigen, was wir in dieser Zeit erreicht haben aber auch wo es noch Verbesserungspotenzial gibt. Ausserdem soll mit dem Kapitel \ref{subsub:Ausblick} aufgezeigt werden, was das weitere Vorgehen für dieses Projekt sein könnte.

\subsubsection{Ergebnisbewertung}
Wie schon in Kapitel \ref{subsub:vergleich-soll-ist} angedeutet, konnten die meisten Ziele erreicht werden. Voraussetzung dafür waren einige Faktoren, unter anderem sehr gute Kommunikation untereinander, die richtig gewählten Management-Methoden und Techniken sowie das Verwenden von einem unterstützenden Toolset. 

Besonders in der Elaboration Phase verhalf das Herunterbrechen der Methode 635 in verschiedene Komponenten dem Verständnis und gab Aufschluss über die Herangehensweise für die technische Umsetzung. 

Diese verlief dann ziemlich effizient, wobei aber immer wieder kleinere Probleme auftauchten. Durch Absprache untereinander und mit dem Betreuer wurde aber immer eine Lösung gefunden. Am Schluss der Entwicklung gab es einige Schwierigkeiten, da der Code mit dem Implementieren weiterer Features immer umfangreicher wurde und somit auch unübersichtlicher. 

Alle erwähnten Punkte kombiniert hatten zur Folge, dass eine lauffähige Applikation zustande kam, die für das Brainstorming verwendbar ist. Auch wenn die Benutzung der Applikation nicht immer ganz klar ist, schränkt sie den Benutzer nicht vor der Kernfunktionalität ein. Dies konnte nur geschehen, weil wir während der Projektzeit ein starker Fokus auf die Funktionalität legten. Die Usability haben wir also gegenüber der Funktionalität abgewogen und darin verhältnismässig weniger Aufwand investiert. 

Das Projekt ist in unseren Augen erfolgreich verlaufen, in einer zukünftigen Arbeit würden beide Autoren ein ähnliches Vorgehen wählen. 

\subsubsection{Bekannte Probleme}
Dieses Kapitel dient dazu, Fehlverhalten in unserem System zu dokumentieren.

\begin{basedescript}{
		\desclabelstyle{\multilinelabel}
		\desclabelwidth{4.5cm}
		\setlength{\itemsep}{5ex}}
	\item [Absturz bei Join Team] Beim erstmaligem Ausführen der Join Team Funktionalität fragt Android nach der Berechtigung für den Zugriff auf die Kamera. Sobald diese gewährleistet wird, stürzt die Applikation aus ungeklärten Gründen ab. Wurde die Berechtigung einmal vergeben, funktioniert das Beitreten des Teams wie gewünscht.
	
	\item [Neue Runde/Overview wird nicht richtig dargestellt] 
	Wenn das Brainstorming gestartet wird oder eine neue Runde beginnt, kommt es vor, dass die Sheets nicht wie gewünscht aktualisiert werden. Abhilfe verschafft die Sync-Funktionlität, die hinter dem Icon versteckt ist.
	
	\item [Kein Feedback beim Team erstellen] 
	Wird zuerst ein Team erstellt, darauf ein Brainstorming Finding erstellt und gestarted und im Anschluss darauf versucht, erneut ein Team zu erstellen, wird man nicht weitergeleitet. Das Team wird zwar auf dem Backend erstellt, jedoch auf dem Frontend nicht aktualisiert.
\end{basedescript}


\subsubsection{Ausblick}
\label{subsub:Ausblick}




