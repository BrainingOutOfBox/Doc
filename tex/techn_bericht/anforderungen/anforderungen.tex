\subsection{Anforderungsspezifikation}

\subsubsection{Funktionale Anforderungen}

\subsubsection{Nicht-Funktionale Anforderungen}
Beim Thema Nicht-Funktionale Anforderungen halten wir uns an die Standards ISO 9126\cite{ISO9126} bzw. dessen Nachfolger ISO 25010\cite{ISO9126_ISO25010}. Beide ISO-Normen sind sich sehr ähnlich und liefern eine gute Checkliste für jegliche Art von Systemanforderungen.

\begin{figure}[h]
	\centering
	\includegraphics[width=1\linewidth]{img/anforderungen/quality}
	\caption[Anforderungskategorien nach ISO 25010]{Anforderungskategorien nach  ISO 25010}
	\label{fig:ISO 25010}
\end{figure}

%TODO Bild Link: https://blog.seibert-media.net/blog/2018/05/14/qualitaet-funktionale-und-nichtfunktionale-anforderungen-in-der-software-entwicklung/

Diese Normen sind sehr umfangreich gestaltet. Wir werden uns daher auf die, für uns, wichtigsten Anforderungen konzentrieren. Um genaue und erfüllbare nicht-funktionale Anforderungen zu definieren, müssen die SMART-Kriterien \cite{SMART} erfüllt sein. 

\begin{description}[leftmargin=!,labelwidth=\widthof{\bfseries Wiederverwendbarkeit}]
	\item[Ressourcennutzung] Die internen Ressourcen Kamera, Dateisystem dürfen nur bei effektivem Bedarf benützt werden. Die CPU-Ressourcen\-nutzung darf im Durchschnitt pro Minute maximal zu 40\% in Anspruch genommen werden.\footnote{Referenzsystem Android: Huawei P10 mit Android Version 8.0.0 mit Hisilicon Kirin 960 CPU und 4GB RAM}\footnote{Referenzsystem iOS: iPhone 6 mit iOS Version 12 mit Dual-core 1.4 GHz Typhoon CPU und 1GB RAM}
	
	\item[Bedienbarkeit] Wenn eine Aktion länger als 1-2s geht, soll dem User ein Wartesymbol angezeigt werden. 
	
	\item[Ästhetik] Die Benutzeroberflächen der Applikation sind so gestaltet, dass die Elemente wiedererkennbar sind (Buttons haben gleichen Stil, leere Textfelder haben Platzhalter). 
	
	\item[Vertraulichkeit] Die Daten einer Brainstorming Session können nur von der zugehörigen Gruppe eingesehen werden. 
	
	\item[Anpassbarkeit] Die Anpassung bestehender oder Integration neuer Brain\-storming-Methoden muss gewährleistet sein.
	
	\item[Installierbarkeit] Die Installation der Applikation auf einem Endgerät erfolgt durch das Ausführen eines *.apk oder *.app. Dieser Prozess soll unter 1 Minute geschehen.
	
	\item[Co-Existenz] Sollte zu einem späteren Zeitpunkt entschieden werden ein Web-Frontend zu programmieren, muss dieses co-existent mit der Xamarin Applikation existieren können.
	
	\item[Wiederherstellbarkeit] Im Falle eines fehlerhaften Features, muss es innerhalb eines Werktages möglich sein, die Applikation wieder auf den letzten funktionierenden Stand zurück zu holen und erneut zu deployen.	
	
	\item[Wiederverwendbarkeit] Die Auswertung von 'Duplicated Code' in SonarQube soll unter 8\% liegen. Dies deutet auf eine hohe Wiederverwendbarkeit hin, denn ansonsten müsste der Code kopiert werden. 
	
	%TODO: Referenz auf Use Cases
	\item[Analysierbarkeit] Das Ausführen eines Use-Cases muss durch Analyse von Logfiles erkennbar sein.
\end{description}
