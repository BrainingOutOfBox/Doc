\subsection{Vorstudie}

\subsubsection{Erste Erfahrungen mit der Methode 635} \label{subsub:erste_erfahrungen_mit_methode_635}
Bevor wir uns mit den technischen Details der Umsetzung zur Cross-Plattform App auseinander gesetzt haben, spielte jeder der beiden Projektmitgliedern die Methode 635, wie in Kapitel \ref{subsec:methode_635_desc} beschrieben, mit seiner Familie, Bekannten oder Freunden einmal durch. 

Dabei wurden folgende persönliche Erfahrungen und Beobachtungen gemacht:

\begin{description}[leftmargin=!,labelwidth=\widthof{\bfseries Interessante Methode}]
	\item[Diskussion gestartet] Schon nach 2-3 Runden konnte beobachtet werden, dass sich spannende Diskussionen zum gestellten Problem entwickelten. Die Teilnehmer mussten sogar angehalten werden die Diskussionen auf dem Papier weiterzuführen, da sonst die Aussagen verloren gehen.
	\item[Interessante Methode] Die Teilnehmer empfanden die Methode 635 als spannend und interessant. Durch die einzelnen Teilnehmer waren auch verschiedenste Blickwinkel auf das Problem vertreten, was wiederum zu unterschiedlichsten neuen Ideen führe.
	\item[Wertung einführen] Was allerdings als kleiner Nachteil empfunden wurde, war der Umstand, dass die Methode 635 in der original Version keine Möglichkeit für Wertungen bietet. 
	
	Als Mensch bildet man sich während des Lesens der einzelnen Ideen automatisch eine Meinung darüber und wird dadurch stark dazu verleitet, eine wertende Bemerkung statt einer neuen oder angepassten Idee zu schreiben.
	
	Eine mögliche Verbesserung könnte darin bestehen, ein Möglichkeit für Wertungen der Ideen einzuführen. Dies könnte so aussehen, dass man neben dem Text auch ein Label (z.B. Pro oder Contra) und die Idee, welche man bewerten will, auswählen kann.
\end{description}

\subsubsection{Xamarin.Forms oder Xamarin native?}
%TODO würde hier die Vorteile und Nachteile von Forms und native zueinander erklären. Die Begründung warum wir uns für das eine entschieden haben, würde ich allerdings ins Architekturentscheide nehmen -> daher auch gleicher Titel im Dokument Architekturentscheide.

\subsubsection{Visual Studio App Center}

\subsubsection{Backend-Technologie}
%TODO würde hier das PlayFramework erklären, was sind die Vorteile und Nachteile zu anderen Systemen etc. Die Begründung warum wir uns dafür entschieden haben, würde ich allerdings ins Architekturentscheide nehmen -> daher auch gleicher Titel im Dokument Architekturentscheide.
