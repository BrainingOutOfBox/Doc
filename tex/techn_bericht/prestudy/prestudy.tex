\subsection{Vorstudie}
In der Vorstudie beschäftigten wir uns zuerst mit der Methode 635 an sich, um ein Gefühl dafür zu bekommen, wie es ist, diese selbst an einem konkreten Problem zu nutzen. 

Weiter beschäftigten wir uns mit Xamarin. Hierbei war vor allem der Entscheid zwischen Xamarin.Forms und Xamarin native von grosser Bedeutung, da dieser im späteren Projektverlauf kaum mehr rückgängig zu machen ist.

Ein weiterer Teil der Vorstudie bestand darin eine passende Umgebung  für das automatische Deployen der Xamarin Applikation zu finden. 

Zum Schluss der Vorstudie beschäftigten wir uns mit der Frage, welches Backend wir für unser Projekt verwenden sollten.

Einige wichtige Punkte und Entscheide waren stark von der Vorstudie abhängig.Es war daher entscheidend, bei der Vorstudie sorgfältig zu arbeiten.


\subsubsection{Erste Erfahrungen mit der Methode 635} \label{subsub:erste_erfahrungen_mit_methode_635}
Bevor wir uns mit den technischen Details der Umsetzung zur Cross-Plattform App auseinander gesetzt haben, spielte jeder der beiden Projektmitgliedern die Methode 635, wie in Kapitel \ref{subsec:methode_635_desc} beschrieben, mit seiner Familie, Bekannten oder Freunden einmal durch. 

Dabei wurden folgende persönliche Erfahrungen und Beobachtungen gemacht:

\begin{description}[leftmargin=!,labelwidth=\widthof{\bfseries Interessante Methode}]
	\item[Diskussion gestartet] Schon nach 2-3 Runden konnte beobachtet werden, dass sich spannende Diskussionen zum gestellten Problem entwickelten. Die Teilnehmer mussten sogar angehalten werden die Diskussionen auf dem Papier weiterzuführen, da sonst die Aussagen verloren gehen.
	\item[Interessante Methode] Die Teilnehmer empfanden die Methode 635 als spannend und interessant. Durch die einzelnen Teilnehmer waren auch verschiedenste Blickwinkel auf das Problem vertreten, was wiederum zu unterschiedlichsten neuen Ideen führe.
	\item[Wertung einführen] Was allerdings als kleiner Nachteil empfunden wurde, war der Umstand, dass die Methode 635 in der original Version keine Möglichkeit für Wertungen bietet. 
	
	Als Mensch bildet man sich während des Lesens der einzelnen Ideen automatisch eine Meinung darüber und wird dadurch stark dazu verleitet, eine wertende Bemerkung statt einer neuen oder angepassten Idee zu schreiben.
	
	Eine mögliche Verbesserung könnte darin bestehen, ein Möglichkeit für Wertungen der Ideen einzuführen. Dies könnte so aussehen, dass man neben dem Text auch ein Label (z.B. Pro oder Contra) und die Idee, welche man bewerten will, auswählen kann.
\end{description}

\subsubsection{Xamarin.Forms oder Xamarin native?}
%TODO würde hier die Vorteile und Nachteile von Forms und native zueinander erklären. Die Begründung warum wir uns für das eine entschieden haben, würde ich allerdings ins Architekturentscheide nehmen -> daher auch gleicher Titel im Dokument Architekturentscheide.

\subsubsection{Visual Studio App Center}


\subsubsection{Backend-Technologie}
Das Play Framework \cite{PlayFramework} ist ein, unter der Apache 2 Lizenz stehendes, Web Application Framework. Es folgt dem MVC-Pattern und erleichtert das Erstellen von Webanwendungen. Durch den Aufbau ermöglicht es dem Entwickler unter anderem auf einfache Art und Weise eine RESTful API zu schreiben. 

Play basiert auf einer zustandslosen und schlanken Architektur. Da es auf Akka aufgebaut ist, nützt es eine vollständig asynchrones I/O. Ausserdem bietet Play minimalen Ressourcenverbrauch. 

Sobald am Code eine Änderung vorgenommen wurde, wird der Server automatisch neugestartet, was die Produktivität des Entwicklers  weiter erhöht.

Jeglicher Code, welcher nicht kompiliert werden konnte, wird zudem im Browser angezeigt und man weiss als Entwickler sofort auf welcher Zeile man nachbessern muss. 

Seit der Version 2.0 des Play Frameworks ist der Framework Core in Scala geschrieben und als Build Tool wird SBT verwendet. Für das Testen stehen verschiedene Test Frameworks sowohl für Java als auch für Scala zur Verfügung. Mittels Scalatest oder JUnit können Unit Tests für beide Sprachen geschrieben werden. Es ist auch möglich Tools wie scoverage für die Code Coverage zu verwenden. 
