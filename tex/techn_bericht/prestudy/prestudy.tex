\subsection{Vorstudie}

\subsubsection{Erste Erfahrungen mit der Methode 635}
Bevor wir uns mit den technischen Details der Umsetzung zur Cross-Plattform App auseinander gesetzt haben, spielte jeder der beiden Projektmitgliedern die Methode 635, wie in Kapitel \ref{subsec:methode_635_desc} beschrieben, mit seiner Familie, Bekannten oder Freunden einmal durch. 

Dabei wurden folgende persönliche Erfahrungen und Beobachtungen gemacht:

\begin{description}[leftmargin=!,labelwidth=\widthof{\bfseries Interessante Methode}]
	\item[Diskussion gestartet] Schon nach 2-3 Runden konnte beobachtet werden, dass sich spannende Diskussionen zum gestellten Problem entwickelten. Die Teilnehmer mussten sogar angehalten werden die Diskussionen auf dem Papier weiterzuführen, da sonst die Aussagen verloren gehen.
	\item[Interessante Methode] Die Teilnehmer empfanden die Methode 635 als spannend und interessant. Durch die einzelnen Teilnehmer waren auch verschiedenste Blickwinkel auf das Problem vertreten, was wiederum zu unterschiedlichsten neuen Ideen führe.
	\item[Labels einführen] Was allerdings als kleiner Nachteil empfunden wurde, war der Umstand, dass die Papierversion keine Möglichkeit für Labels bietet. Zum Teil war nämlich nicht ganz klar, ob ein Nachtrag positiv auf eine beschriebene Lösung gemeint war oder negativ.
\end{description}


\subsubsection{Xamarin.Forms oder Xamarin native}

\subsubsection{Visual Studio App Center}

\subsubsection{Backend-Technologie}

\subsubsection{Methode 635 als verteiltes System}
%TODO erwähnen, dass wir über ein verteiltes System nachgedacht haben? 
%Nach kurzem Gespräch mit Thomas Bocek hat er davon abgeraten. Ein verteiltes System sei immer komplexer und komplizierter als ein Server/Client System. Für diese geringe Anzahl von Teilnehmern, welche prinzipiell nur Messages austauschen, lohnt es sich nicht ein verteiltes System zu bauen. 