\subsection{Ergebnisse}
Das Kapitel der Ergebnisse befasst sich mit der konkreten Umsetzung der Xamarin Applikation und dem PlayFramework. Dabei werden die wichtigsten Funktionen der jeweiligen Bestandteile genauer erklärt und beim Xamarin App zusätzlich noch eine kurze Retrospektive auf geplantes Design und effektiver Umsetzung gegeben.

Zur besseren Übersicht wurde der Code vereinzelt gekürzt. Dies ist durch 3 Punkte (...) gekennzeichnet.

\subsubsection{Implementierung des PlayFrameworks}
Die Umsetzung des Backends basiert auf dem PlayFramework. Die Gründe für diesen Entscheid kann in Kapitel \ref{vorstudie} nachgelesen werden. 

Um die Daten permanent zu speichern, besitzt das PlayFramework eine Anbindung an eine MongoDB Instanz. Die Anbindung wurde mit dem MongoDB Async Driver für Java \cite{MongoDBAsyncDriver} realisiert. Die Implementation ist daher auch in Java geschrieben. Für die Dokumentation des Backends haben wir Swagger \cite{swagger} verwendet.

\paragraph*{ParticipantController.java}

\lstset{language=JAVA, showstringspaces=false, frame=single, captionpos=b, label=createParticipant, breaklines=true, numbers=left}
\begin{lstlisting}[caption={Participant erstellen}]
public Result createParticipant(){

    JsonNode body = request().body().asJson();

    if (body == null ) {
        return forbidden(Json.toJson(new ErrorMessage("Error", "json body is null")));
    } else if(  body.hasNonNull("username") &&
            	body.hasNonNull("password") &&
            	body.hasNonNull("firstname") &&
            	body.hasNonNull("lastname")) {

        Participant participant = new Participant(body.get("username").asText(), body.get("password").asText(), body.get("firstname").asText(), body.get("lastname").asText());

        participantCollection.insertOne(participant, new SingleResultCallback<Void>() {
            @Override
            public void onResult(Void result, Throwable t) {
                Logger.info("Inserted Participant!");
            }
        });

        return ok(Json.toJson(new SuccessMessage("Success", "Participant successfully inserted")));
    }

    return forbidden(Json.toJson(new ErrorMessage("Error", "json body not as expected")));
}
\end{lstlisting}

Bei der Methode für das Erstellen von Participants, wird zuerst geprüft, ob ein HTML-Body existiert. Ist dies nicht der Fall, wird ein HTTP-Response Status-Code (Zeile 24) zurückgesendet.

Existiert ein HTML-Body, wird dieser auf die Existenz der Felder \textit{username}, \textit{password}, \textit{firstname} und \textit{lastname} geprüft (Zeile 7-10). Daraus wird als nächstes ein \texttt{participant} erstellt (Zeile 12) und mittels  \texttt{insertOne} in die \texttt{participant\-Collection} gespeichert (Zeile 14).

Zulezt wird der Status-Code 200 mit einer Nachricht an den Absender zurück gesendet.

\begin{lstlisting}[caption={Login}]
public Result login() throws UnsupportedEncodingException, ExecutionException, InterruptedException {
...
if (body.hasNonNull("username") && body.hasNonNull("password")) {
    CompletableFuture<Participant> future = new CompletableFuture<>();

    participantCollection.find(and(
    eq("username", body.get("username").asText()),
    eq("password", body.get("password").asText()))).first(new SingleResultCallback<Participant>() {
        @Override
        public void onResult(Participant participant, Throwable t) {
            if (participant != null) {
                Logger.info("Found participant");
                future.complete(participant);
            } else {
                future.complete(null);
            }
        }
    });

    if (future.get()!= null){
        ObjectNode result = Json.newObject();
        result.putPOJO("participant", future.get());
        result.put("access_token", getSignedToken(7l));
        return ok(result);
    } else {
       ...
    }

} else {
    ...
}

}
\end{lstlisting}

Für das Login eines Participants wird auch hier zuerst nach der Existenz der Felder \textit{username} und \textit{password} im HTML-Body geschaut (Zeile 3). Im nächsten Schritt wird die Datenbank nach einem Participant mit den angegebenen Werten durchsucht (Zeile 6-8). 

Da wir einen asynchronen Treiber verwenden, benötigen wir ein \texttt{CompletableFuture}, um das Resultat der Abfrage darin zu speichern und um mittels \texttt{future.get()} darauf zugreifen zu können (Zeile 20).

Am Ende wird neben dem gefundenen \textit{participant} noch ein \textit{JWT-Token} dem Resultat angefügt (Zeile 21-24).

\paragraph*{TeamController.java}
\begin{lstlisting}[caption={Einem Team beitreten}]
public Result joinBrainstormingTeam(String teamIdentifier) throws ExecutionException, InterruptedException {
...
if (brainstormingTeam!= null && brainstormingTeam.getNrOfParticipants() > brainstormingTeam.getCurrentNrOfParticipants() && brainstormingTeam.joinTeam(participant)) {

    teamCollection.updateOne(eq("identifier", teamIdentifier),combine(set("participants", brainstormingTeam.getParticipants()), inc("currentNrOfParticipants", 1)), new SingleResultCallback<UpdateResult>() {
        @Override
        public void onResult(final UpdateResult result, final Throwable t) {
            Logger.info(result.getModifiedCount() + " Team successfully updated");
        }
    });

    return ok(Json.toJson(new SuccessMessage("Success", "Participant successfully added to the brainstormingTeam")));

} else {
    ...
}
...
}
\end{lstlisting}

Dieses Beispiel soll zeigen, wie mit dem MongoDB Async Driver ein Eintrag mittels \texttt{updateOne} aktualisiert werden kann.

Wie auch schon bei der \texttt{find} Methode, kennzeichnet der erste Parameter das Dokument, welches man aktualisieren möchte. Der zweite Parameter steht für die Felder und deren neuen Werte und der dritte und letzte Parameter ist wieder der \texttt{SingleResultCallback} und beschreibt wie das Resultat weiter prozessiert wird. All dies ist auf Zeile 5-8 zu finden.

\paragraph*{FindingController.java}
\begin{lstlisting}[caption={Watcher für BrainstormingFinding}]
private void startWatcherForBrainstormingFinding(String identifier){

ScheduledExecutorService executor = Executors.newSingleThreadScheduledExecutor();

TimerTask task = new TimerTask() {
    @Override
    public void run() {
        try {
    	...
	if (finding.getCurrentRound() > finding.getBrainsheets().size()){
	    lastRound(identifier);
	    executor.shutdown();
	}

	if (endDateTime.plusSeconds(30).isBeforeNow() ||
	finding.getDeliveredBrainsheetsInCurrentRound() >= finding.getBrainsheets().size()){
	nextRound(identifier);
	}
            cancel();

        } catch (ExecutionException e) {
            e.printStackTrace();
        } catch (InterruptedException e) {
            e.printStackTrace();
        }
    }
};

executor.scheduleAtFixedRate(task, 1000L, 5000L, TimeUnit.MILLISECONDS);
}
\end{lstlisting}

Um den Zustand über die verbleibende Zeit oder die bereits eingereichten \textit{Brainsheets} überwachen zu können, setzen wir einen \texttt{ScheduledExecutorService} \cite{JavaTimer} ein. Dieser erlaubt uns alle 5000ms bzw. alle 5s den \texttt{TimerTask} auf Zeile 5 auszuführen. 

Der \texttt{TimerTask} prüft zuerst, ob die aktuelle Runde schon die letzte Runde ist (Zeile 10). Ist dies der Fall, führt er die Methode \textit{lastRound(identifier)} aus und beendet den executor, sodass keine neuen \texttt{TimerTask} Objekte gestartet werden. In diesem Zustand ist das gesamte \textit{BrainstormingFinding} ausgefüllt.

Ist dies nicht der Fall, prüft er als nächstes, ob die Endzeit der aktuellen Runde plus 30s noch vor der aktuellen Uhrzeit liegt (Zeile 15). Die Bedingung auf Zeile 16 prüft, ob alle \textit{Brainsheets}, welche für die Abgabe erwartet werden schon abgegeben wurden. Unabhängig welche dieser zwei Bedingungen (Zeile 15 oder 16) zuerst eintrifft, es wird anschliessend immer die Methode \textit{nextRound(identifier)} ausgeführt.

Sollte keine der Bedingungen (Zeile 10, 15 oder 16) zutreffen, so beendet sich der \texttt{TimerTask} mittels \textit{cancel} selbst. 

Da der executor aber nach 5s den nächsten \texttt{TimerTask} startet, ist so für die gesamte Dauer, für die das \textit{BrainstormingFinding} läuft, ein 'Watcher' für den korrekten Ablauf zuständig. Die Methode \textit{startWatcherForBrainstormingFinding(String identifier)} wird beim Start eines \textit{BrainstormingFinding} ausgeführt.

\begin{lstlisting}
public Result startBrainstorming(String findingIdentifer) throws ExecutionException, InterruptedException {
        startWatcherForBrainstormingFinding(findingIdentifer);
        return nextRound(findingIdentifer);
}
\end{lstlisting}
\subsubsection{Implementierung der Xamarin App}