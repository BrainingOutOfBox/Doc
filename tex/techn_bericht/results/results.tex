\subsection{Ergebnisse}
Das Kapitel der Ergebnisse befasst sich mit der konkreten Umsetzung der Xamarin Applikation und dem PlayFramework. Wir haben uns dabei ganz konkret für die nachfolgenden Code-Beispiele entschieden. Der Grund dafür ist, dass diese entweder den typischen Aufbau einer Methode zeigen oder wie das Listing \ref{watcher}, eine Kernlogik darstellen. 
Bei den Ergebnissen der Xamarin App zeigen wir zusätzlich noch eine kurze Retrospektive auf geplantes Design und effektiver Umsetzung des GUIs.

Zur besseren Übersicht wurde der Code vereinzelt gekürzt. Dies ist durch 3 Punkte (...) gekennzeichnet.

\subsubsection{Implementierung des PlayFrameworks}
Die Umsetzung des Backends basiert auf dem PlayFramework. Die Gründe für diesen Entscheid können in Kapitel \ref{vorstudie} nachgelesen werden. 

Um die Daten permanent zu speichern, besitzt das PlayFramework eine Anbindung an eine MongoDB Instanz. Die Anbindung wurde mit dem MongoDB Async Driver für Java \cite{MongoDBAsyncDriver} realisiert. Die Implementation ist daher auch in Java geschrieben. Für die Dokumentation des Backends verwendeten wir Swagger \cite{swagger}.

\paragraph*{ParticipantController.java}
Die nachfolgenden Listings zeigen einen Ausschnitt aus der \texttt{ParticipantController.java} Klasse. Diese befindet sich in der LogicComponent (Abbildung \ref{fig:architektur-methode635}) vom PlayFramework.

\lstset{language=JAVA, showstringspaces=false, frame=single, captionpos=b, label=createParticipant, breaklines=true, numbers=left}
\begin{lstlisting}[caption={Participant erstellen}, label=participantErstellen]
public Result createParticipant(){

    JsonNode body = request().body().asJson();

    if (body == null ) {
        return forbidden(Json.toJson(new ErrorMessage("Error", "json body is null")));
    } else if(  body.hasNonNull("username") &&
            	body.hasNonNull("password") &&
            	body.hasNonNull("firstname") &&
            	body.hasNonNull("lastname")) {

    Participant participant = new Participant(body.get("username").asText(), body.get("password").asText(), body.get("firstname").asText(), body.get("lastname").asText());

    participantCollection.insertOne(participant, new SingleResultCallback<Void>() {
        @Override
        public void onResult(Void result, Throwable t) {
            Logger.info("Inserted Participant!");
        }
    });

    return ok(Json.toJson(new SuccessMessage("Success", "Participant successfully inserted")));
    }

    return forbidden(Json.toJson(new ErrorMessage("Error", "json body not as expected")));
}
\end{lstlisting}

Bei der Methode für das Erstellen von Participants, prüft das Framework zuerst, ob ein HTML-Body existiert. Ist dies nicht der Fall, sendet es ein HTTP-Response Status-Code (Zeile 24) zurück.

Existiert ein HTML-Body, wird dieser auf die Existenz der Felder \textit{username}, \textit{password}, \textit{firstname} und \textit{lastname} geprüft (Zeile 7-10). Daraus erstellt das Framework als nächstes ein \texttt{participant} (Zeile 12), welcher mittels  \texttt{insertOne} in die \texttt{participant\-Collection} gespeichert wird (Zeile 14).

Zuletzt sendet das PlayFramework eine Antwort mit dem HTTP-Status Code 200 und einer Nachricht an den Absender. 

\begin{lstlisting}[caption={Login}]
public Result login() throws UnsupportedEncodingException, ExecutionException, InterruptedException {
...
if (body.hasNonNull("username") && body.hasNonNull("password")) {
    CompletableFuture<Participant> future = new CompletableFuture<>();

    participantCollection.find(and(
    eq("username", body.get("username").asText()),
    eq("password", body.get("password").asText()))).first(new SingleResultCallback<Participant>() {
        @Override
        public void onResult(Participant participant, Throwable t) {
            if (participant != null) {
                Logger.info("Found participant");
                future.complete(participant);
            } else {
                future.complete(null);
            }
        }
    });

    if (future.get()!= null){
        ObjectNode result = Json.newObject();
        result.putPOJO("participant", future.get());
        result.put("access_token", getSignedToken(7l));
        return ok(result);
    } else {...}
} else {...}
}
\end{lstlisting}

Für das Login eines Participants schaut auch hier zuerst das Framework im HTML-Body nach der Existenz der Felder \textit{username} und \textit{password} (Zeile 3). Im nächsten Schritt durchsucht es die Datenbank nach einem Participant mit den angegebenen Werten (Zeile 6-8). 

Da wir einen asynchronen Treiber verwenden, benötigen wir ein \texttt{CompletableFuture}, um das Resultat der Abfrage darin abzuspeichern und um mittels \texttt{future.get()} darauf zugreifen zu können (Zeile 20).

Am Ende wird dem Resultat neben dem gefundenen \textit{participant} noch ein \textit{JWT-Token} angefügt (Zeile 21-24).

\paragraph*{TeamController.java}
Das Listing \ref{teamBeitreten} zeigt einen Ausschnitt aus der \texttt{TeamController.java} Klasse. Auch diese befindet sich in der LogicComponent (Abbildung \ref{fig:architektur-methode635}) vom PlayFramework.
\begin{lstlisting}[caption={Einem Team beitreten}, label=teamBeitreten]
public Result joinBrainstormingTeam(String teamIdentifier) throws ExecutionException, InterruptedException {
...
if (brainstormingTeam!= null && brainstormingTeam.getNrOfParticipants() > brainstormingTeam.getCurrentNrOfParticipants() && brainstormingTeam.joinTeam(participant)) {

    teamCollection.updateOne(eq("identifier", teamIdentifier),combine(set("participants", brainstormingTeam.getParticipants()), inc("currentNrOfParticipants", 1)), new SingleResultCallback<UpdateResult>() {
        @Override
        public void onResult(final UpdateResult result, final Throwable t) {
            Logger.info(result.getModifiedCount() + " Team successfully updated");
        }
    });

    return ok(Json.toJson(new SuccessMessage("Success", "Participant successfully added to the brainstormingTeam")));

} else {
    ...
}
...
}
\end{lstlisting}

Dieses Beispiel soll zeigen, wie mit dem MongoDB Async Driver ein Eintrag mittels \texttt{updateOne} aktualisiert werden kann.

Wie auch schon bei der \texttt{find} Methode, kennzeichnet der erste Parameter das Dokument, welches man aktualisieren möchte. Der zweite Parameter steht für die Felder und deren neuen Werte und der dritte und letzte Parameter ist wieder der \texttt{SingleResultCallback} und beschreibt wie das Resultat weiter prozessiert wird. All dies ist auf Zeile 5-8 zu finden.

\paragraph*{FindingController.java}
Das Listing \ref{watcher} zeigt einen Ausschnitt aus der \texttt{FindingController.java} Klasse. Wie auch schon die vorherigen Klassen, befindet sich auch diese in der LogicComponent (Abbildung \ref{fig:architektur-methode635}) vom PlayFramework.
\begin{lstlisting}[caption={Watcher für BrainstormingFinding}, label=watcher]
private void startWatcherForBrainstormingFinding(String identifier){

ScheduledExecutorService executor = Executors.newSingleThreadScheduledExecutor();

TimerTask task = new TimerTask() {
    @Override
    public void run() {
        try {
    	...
	if (finding.getCurrentRound() > finding.getBrainsheets().size()){
	    lastRound(identifier);
	    executor.shutdown();
	}

	if (endDateTime.plusSeconds(30).isBeforeNow() ||
	finding.getDeliveredBrainsheetsInCurrentRound() >= finding.getBrainsheets().size()){
	nextRound(identifier);
	}
            cancel();

        } catch (ExecutionException e) {
            e.printStackTrace();
        } catch (InterruptedException e) {
            e.printStackTrace();
        }
    }
};

executor.scheduleAtFixedRate(task, 1000L, 5000L, TimeUnit.MILLISECONDS);
}
\end{lstlisting}

Um den Zustand über die verbleibende Zeit oder die bereits eingereichten \textit{Brainsheets} überwachen zu können, setzen wir einen \texttt{ScheduledExecutorService} \cite{JavaTimer} ein. Dieser erlaubt uns alle 5000ms bzw. alle 5s den \texttt{TimerTask} auf Zeile 5 auszuführen. 

Der \texttt{TimerTask} prüft zuerst, ob die aktuelle Runde schon die letzte Runde ist (Zeile 10). Ist dies der Fall, führt er die Methode \textit{lastRound(identifier)} aus und beendet den executor, sodass keine neuen \texttt{TimerTask} Objekte gestartet werden. In diesem Zustand ist das gesamte \textit{BrainstormingFinding} ausgefüllt.

Ist dies nicht der Fall, prüft er als nächstes, ob die Endzeit der aktuellen Runde plus 30s noch vor der aktuellen Uhrzeit liegt (Zeile 15). Die Bedingung auf Zeile 16 prüft, ob alle \textit{Brainsheets}, welche für die Abgabe erwartet werden schon abgegeben wurden. Unabhängig welche dieser zwei Bedingungen (Zeile 15 oder 16) zuerst eintrifft, es wird anschliessend immer die Methode \textit{nextRound(identifier)} ausgeführt.

Sollte keine der Bedingungen (Zeile 10, 15 oder 16) zutreffen, so beendet sich der \texttt{TimerTask} mittels \textit{cancel} selbst. 

Da der executor aber nach 5s den nächsten \texttt{TimerTask} startet, ist so für die gesamte Dauer, für die das \textit{BrainstormingFinding} läuft, ein 'Watcher' für den korrekten Ablauf zuständig. Die Methode \textit{startWatcherForBrainstormingFinding(String identifier)} wird beim Start eines \textit{BrainstormingFinding} ausgeführt.

\begin{lstlisting}
public Result startBrainstorming(String findingIdentifer) throws ExecutionException, InterruptedException {
        startWatcherForBrainstormingFinding(findingIdentifer);
        return nextRound(findingIdentifer);
}
\end{lstlisting}

\subsubsection{Implementierung der Xamarin App}