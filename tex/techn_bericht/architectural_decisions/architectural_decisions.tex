\subsection{Architekturentscheide}
Wesentliche Entscheide, welche wir während dem Projekt getroffen haben, sind hier detailliert begründet. Auch Gedanken oder Ideen, welche wir während der Analysephase hatten, dann aber verworfen haben, sind hier aufgeschrieben.

\subsubsection{Erste Erfahrungen mit der Methode 635}
Wie in Kapitel \ref{subsub:erste_erfahrungen_mit_methode_635} beschrieben, tendiert der Mensch dazu, die Ideen der anderen Teilnehmer automatisch zu bewerten. Als mögliche Lösung wurde die Integration einer Bewertungsmöglichkeit beschrieben.


Wir haben uns allerdings darauf geeinigt, dass wir die originale Version, also ohne die Möglichkeit für eine Wertung, als Vorlage nehmen und diese auch so in unserer Cross-Plattform Applikation umsetzen. 


Die Integration einer Bewertungsmöglichkeit wird als optionales Feature angesehen und lediglich bei genügend Restzeit im Projekt umgesetzt.

\subsubsection{Xamarin.Forms oder Xamarin native}\label{subsubsec:forms-vs-native}
Für diesen Entscheid galt es zu evaluieren, welche User Controls für unsere Applikation die exotischsten sind. Dies, weil Xamarin.Forms eine Menge an Standard-Controls anbietet, die vom Framework selber in das jeweilige Betriebssystem konvertiert werden. Sind alle vorgesehenen Benutzerelemente in Forms enthalten, sparen wir uns die Zeit, betriebssystemspezifische Elemente zu entwickeln. 

Für unser Projekt haben wir folgende Benutzerelemente als exotisch oder kritisch definiert:
\begin{itemize}
	\item Canvas Control für Zeichnen einer Idee
	\item Camera Funktion für das Erkennen von Quick Response-Codes (QR-Codes)
	\item Verarbeitung und Generierung von QR-Codes
\end{itemize}

Nach einer Recherche stellte sich heraus, dass sich ein Canvas View von Google namens SkiaSharp \cite{skiaSharp} eignet. Darauf lässt sich gemäss der Dokumentation zeichnen sowie definierte Formen einfügen. Dies könnte auch für eine Erweiterung spannend sein, in der Patterns in UML als Vorlage angeboten werden können.

Für die Kamera-Funktionalität steht ein NuGet-Packet (Xam.Media.Plugin \cite{xam-media-plugin}) bereit, das uns diese Arbeit abnehmen wird.

Das Generieren und Lesen der QR-Codes ist an sich kein Problem von Xamarin.Forms, denn grundsätzlich müssen die von der Kamera generierten Files eingelesen und ins entsprechende QR-Code-Tool eingefügt werden. Hierfür eignet sich das NuGet ZXing.Net \cite{zxing.net}. 

Es stellte sich relativ rasch heraus, dass die gewünschten Funktionalitäten in Xamarin.Forms in ausreichender Qualität enthalten sind und uns das individuelle Entwickeln dadurch abgenommen wird.

\subsubsection{Backend-Technologie}
Neben den Vorteilen, wie der schlanken und zustandslosen Architektur des Play\-Frameworks, dem asynchronen und nicht-blockierenden Verhalten und den vielen unterstützten Bibliotheken, haben wir uns hauptsächlich dafür entschieden, weil wir in anderen Projekten schon sehr gute Erfahrungen mit dem PlayFramework gemacht haben.

Ein weiterer Grund bestand darin, dass das PlayFramework nicht nur in Scala sondern auch in Java geschrieben ist. Mit Java kennen wir uns beide gut aus und mussten uns so keine neue Programmiersprache aneignen.

Da wir uns schon relativ früh für eine MongoDB als Datenbanksystem entschieden hatten, viel die Wahl für das PlayFramework erst recht, als wir einen asynchronen MongoDB-Treiber für Java gefunden hatten.

\subsubsection{MongoDB als Datenbanksystem}
MongoDB (abgeleitet von humongous) ist eine dokumentorientierte, einfache, dynamische und skalierbare NoSQL Datenbank , welche von der MongoDB Inc. entwickelt wird \cite{MonogDBWikipedia} \cite{MonogDBDZone} \cite{MonogDB}. 

Die Basis für das Speichern von Informationen bilden die sogenannten Documents. Die Datenobjekte werden in separaten Documents innerhalb einer Collection (anders als bei traditionellen relationalen Datenbanken in Spalten und Zeilen) gespeichert \cite{MonogDBDZone}. Mit den Documents können zudem  hierarchische Strukturen und Relationen sehr einfach gespeichert werden.

Der Vorteil einer MongoDB Datenbank liegt darin, dass zusammengehörige Informationen gemeinsam in einem Document gespeichert werden. Dies ermöglicht einen schnellen Zugriff auf die Daten mittels der MongoDB Query Language. Da MongoDB zudem ohne Schemas auskommt, ist es nicht nötig, die Datenbank offline zu nehmen, wenn man ein neues Feld einfügen möchte \cite{MonogDB}.

Weitere Vorteile sind laut \href{DZone.com}{DZone.com} die hohe Performance, Verfügbarkeit (durch Replikas) und Skalierung (durch Sharding). Da alle Informationen zusammen in einem Document gespeichert sind, sind auch keine Joins zu anderen Tabellen notwendig. Auch unterstützt MongoDB Funktionen für die Speicherung von Geoinformationen \cite{MonogDBDZone}.

Der Hauptgrund warum wir uns für MongoDB als Datenbanksystem entschieden haben, war die Tatsache, dass alle zusammengehörigen Informationen in einem Document abgelegt werden können. Auch die Möglichkeit hierarchische Strukturen (bei uns die \textit{Brainsheets}, \textit{Brainwaves} und \textit{Ideas}) einfach zu Speichern, hat uns viel Zeit erspart.

Die Mächtigkeit von relationalen Datenbanken im Bereich einer Auswertung, ist in unserem Projekt nicht notwendig. Daher haben wir auch von Beginn an auf das Paradigma der dokumentorientieren Datenbanken gesetzt. Grund warum wir uns schlussendlich für MongoDB und nicht für eine andere dokumentorientierte Datenbank entschieden haben ist, dass MongoDB Thema während dem Studium ist und wir schon einige Erfahrung damit hatten.

\subsubsection{Methode 635 als Peer-to-Peer-System}
Wir haben uns auch überlegt, die Cross-Plattform Applikation d.h. vor allem die Kommunikation zwischen den einzelnen Teilnehmer, als Peer-to-Peer System \cite{Peer2Peer} zu konzipieren.


Prof. Thomas Bocek, Professor für verteilte Systeme an der Hochschule Rapperswil, hat uns allerdings davon abgeraten. Ein verteiltes System sei immer komplexer und komplizierter als ein Server/Client System. Für diese geringe Anzahl von Teilnehmern, welche prinzipiell nur Messages austauschen, lohnt es sich nicht ein verteiltes System zu bauen. 


Daher haben wir uns für eine klassische Server/Client Architektur entschieden. Andere Varianten der Kommunikation, wie Webhooks oder Websockets werden daher nicht weiter verfolgt.

\subsubsection{Kommunikation zwischen Server und App}
Die Kommunikation zwischen dem zentralen Server und den Clients, also den Cross-Plattform Applikationen in unserem Fall, ist von grosser Bedeutung. Wegen der Problematik der Network Address Translation kurz NAT \cite{NAT} kann diese prinzipiell auf zwei Arten erfolgen: Entweder man verwendet Websockets \cite{WebSockets}, welche eine permanente Verbindung zwischen Server und Client öffnen oder die Kommunikation beginnt ausschliesslich beim Client. 


Wir haben uns für das stetige Abfragen von Informationen (Polling) entschieden, da es eine einfache Variante darstellt. Zwar werden dadurch vermeidbare Requests an den Server gesendet, da aber die Anzahl an Teilnehmer bzw. die Ressourcennutzung des Backends in einem vertretbaren Rahmen liegt, ist diese Polling-Variante völlig in Ordnung.  

\textbf{Zwei Beispiele für Polling in der Applikation}: Wenn ein Teilnehmer seine Ideen aufschreibt, fragt die Applikation den Server im Hintergrund in regelmässigen Abständen nach der verbleibenden Zeit für diese Runde ab.


Auch nach der Abgabe der aufgeschriebenen Ideen an den Server, muss der Teilnehmer warten, bis er die Ideen bzw. das Blatt seines Nachbarn bekommt. Dafür muss die Applikation immer wieder den Server fragen, ob der Nachbar überhaupt sein Blatt bzw. seine Ideen abgegeben hat. Bevor dies nicht geschehen ist, kann der Teilnehmer auch nicht weitermachen.


\textbf{Begründung in Zahlen:}
Bei diesen beiden Szenarien sendet die Applikation pro Sekunde einen Request an den Server. Wenn wir bei der Standardmethode von 6 Teilnehmern bleiben, wären das 6 Request pro Sekunde. Wenn wir nun noch annehmen, dass 50 gleichzeitige Ausführungen stattfinden, ergibt das 300 Requests pro Sekunde für eines der beiden Szenarien.


Laut den Release-Informationen zur Version 2.5 von Play \cite{Play25} ist das PlayFramework in der Lage 60'000 Requests pro Sekunde zu verarbeiten.


Nehmen wir nun diese 60'000 Requests als Basis für unsere Berechnung, entsprechen unsere 300 Requests gerade einmal 0.5\% der möglichen 60'000 Requests. Für beide Szenarien entsprechend das Doppelte, also 1\%.

\[\frac{300Requests}{1s}=\frac{1Request}{1s}\cdot6Participants \cdot50Brainstormings\]
\[0.5\%= \frac{\frac{300Requests}{1s} \cdot 100}{\frac{60000Requests}{1s}} = \frac{300 \cdot 100}{60000}\]

\textbf{Fazit:} Diese kleine Rechnung verdeutlich schon ziemlich stark, dass die Variante mit dem Polling das PlayFramework in keinster Weise an dessen Limit bringt. Selbst wenn die Anzahl an Teilnehmern oder gleichzeitigen Ausführungen erhöht wird, ist das PlayFramework im Stande die eingehenden Requests noch zu verarbeiten. Auch andere gleichzeitige Aufrufe an das PlayFramework können dann noch ausgeführt werden.

Sollte es dennoch jemals zu einer zu starken Ressourcennutzung durch das Polling kommen, könnte der Algorithmus so angepasst werden, dass dieser nicht jede Sekunde das Backend z.B. nach der verbleibenden Zeit abfragt sondern bei viel verbleibender Zeit nur alle 30 Sekunden und bei wenig verbleibender Zeit jede Sekunde.

Auf diese Weise kann die Anzahl an Requests drastisch reduziert werden.
