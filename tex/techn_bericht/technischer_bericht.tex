\section{Technischer Bericht}

\subsection{Einleitung und Übersicht}


\subsection{Methode 635}
Die gesamte Beschreibung der Methode 635 wurde von \href{https://kreativitätstechniken.info/6-3-5-methode/}{kreativitästechniken.info} übernommen.


Die 6-3-5-Methode (auch Methode 635 genannt) ist eine Brainwriting-Kre\-ati\-vi\-täts\-technik. Der Name der Methode leitet sich aus den drei wesentlichen Eigenschaften der Methode ab: 6 Teilnehmer erhalten jeweils ein Blatt, auf dem sie 3 Ideen notieren und die Blätter dann insgesamt 5 mal weiterreichen.

\subsubsection*{Anwendungsgebiete der Methode 635}
Die 6-3-5-Methode ist eine Variante des Brainwriting. Sie eignet sich für die erste Phase im kreativen Prozess. Mit ihrer Hilfe werden Ideen gesammelt ohne dass eine Bewertung stattfindet. In kurzer Zeit können im Idealfall 108 Ideen entstehen. Die Aufforderung, bestehende Ideen aufzugreifen und weiterzuentwickeln macht die 6-3-5-Methode zu einer konstruktiven Kreativitätstechnik. Gleichzeitig kann die strukturierte Form jedoch auch die Kreativität bremsen. In der Praxis entstehen daher oft etwas weniger Ideen.

\subsubsection*{Vorgehen bei der Methode 635}
Der Moderator erklärt zunächst die Regeln der 6-3-5-Methode, führt die Teilnehmer in das Ausgangsproblem ein und ist im Folgenden für die Zeitmessung verantwortlich. Sobald die Teilnehmer über die Ausgangsfrage oder -problem aufgeklärt sind, startet die erste von sechs Runden. In jeder Runde werden die Teilnehmer aufgerufen, die oberste noch freie Zeile, bestehend aus 3 Kästchen, mit ihren Ideen zu füllen. Dabei sollten die Teilnehmer die Ideen der Vorgänger aufgreifen, erweitern und/oder weiterentwickeln. Nach einer festgelegten Zeit von beispielsweise 5 Minuten beendet der Moderator die Runde. Die Teilnehmer reichen ihr Arbeitsblatt im Uhrzeigersinn an ihren Sitznachbarn weiter und eine neue Runde beginnt. Im Idealfall sind nach 6 Runden genau 6*18 = 108 Ideen entstanden. In der Praxis ist die Anzahl aufgrund von doppelten oder leeren Einträgen wahrscheinlich etwas geringer. Dennoch sollten nun zahlreiche Ideen vorliegen.


Nun kann eine Diskussion, Analyse und Bewertung der Ideen erfolgen.

\newpage

% Hier soll beschrieben werden wie der technische Bericht aufgebaut ist und was die Kapitel beinhalten.

\subsection{Vorstudie}

\subsubsection{Xamarin.Forms oder Xamarin native}

\subsubsection{Visual Studio App Center}

\subsubsection{Backend-Technologie}

\subsubsection{Methode 635 als verteiltes System}
%TODO erwähnen, dass wir über ein verteiltes System nachgedacht haben? 
%Nach kurzem Gespräch mit Thomas Bocek hat er davon abgeraten. Ein verteiltes System sei immer komplexer und komplizierter als ein Server/Client System. Für diese geringe Anzahl von Teilnehmern, welche prinzipiell nur Messages austauschen, lohnt es sich nicht ein verteiltes System zu bauen. 
\newpage

\subsection{Anforderungsspezifikation}

\subsubsection{Funktionale Anforderungen}

\subsubsection{Nicht-Funktionale Anforderungen}
Beim Thema Nicht-Funktionale Anforderungen halten wir uns an die Standards ISO 9126\cite{ISO9126} bzw. dessen Nachfolger ISO 25010\cite{ISO9126_ISO25010}. Beide ISO-Normen sind sich sehr ähnlich und liefern eine gute Checkliste für jegliche Art von Systemanforderungen.

\begin{figure}[h]
	\centering
	\includegraphics[width=1\linewidth]{img/anforderungen/quality}
	\caption[Anforderungskategorien nach ISO 25010]{Anforderungskategorien nach  ISO 25010}
	\label{fig:ISO 25010}
\end{figure}

%TODO Bild Link: https://blog.seibert-media.net/blog/2018/05/14/qualitaet-funktionale-und-nichtfunktionale-anforderungen-in-der-software-entwicklung/

Diese Normen sind sehr umfangreich gestaltet. Wir werden uns daher auf die, für uns, wichtigsten Anforderungen konzentrieren.

\begin{description}[leftmargin=!,labelwidth=\widthof{\bfseries Wiederherstellbarkeit}]
	\item[Ressourcennutzung] Die internen Ressourcen wie Kamera, Dateisystem, Batterie usw. müssen sparsam und dürfen nur falls nötig eingesetzt werden.
	\item[Bedienbarkeit] Die Bedienbarkeit der Applikation muss einfach und intuitiv gestaltet sein.
	\item[Fehlervermeidung] Dem Benutzer müssen im Falle eines Fehler einfache und verständliche Fehlermeldungen präsentiert werden.
	\item[Ästhetik] Die Benutzeroberflächen der Applikation dürfen sich nicht wesentlich von den in Xamarin verwendetet Formen unterscheiden.
	\item[Vertraulichkeit] Die Daten der Endbenutzer müssen zu jedem Zeitpunkt vertraulich behandelt werden.
	\item[Anpassbarkeit] Die Anpassung oder Integration von neuen Brainstorming-Methoden muss gewährleistet sein.
	\item[Installierbarkeit] Die Installation der Applikation auf einem Endgerät muss einfach gestaltet sein.
	\item[Co-Existenz] Sollte zu einem späteren Zeitpunkt entschieden werden ein Web-Frontend zu programmieren, muss dieses co-existent mit der Xamarin Applikation existieren können.
	\item[Wiederherstellbarkeit] Im Falle eines fehlerhaften Features, muss es innerhalb eines Werktages möglich sein, die Applikation wieder auf den alten Stand zurück zu holen und erneut zu deployen.	
	\item[Wiederverwendbarkeit] Es ist darauf zu achten, dass der geschriebene Code, falls möglich, wiederverwendet wird.
	\item[Analysierbarkeit] Es muss möglich sein, die Endnutzer und deren Verhalten mit der Applikation zu analysieren.	
\end{description}

\newpage

\input{./tex/techn_bericht/domain_analyse/domain_analyse}
\newpage


\subsection{Ergebnisse}

\subsection{Schlussfolgerungen}
\subsubsection{Ergebnisbewertung}
\subsubsection{Ausblick}