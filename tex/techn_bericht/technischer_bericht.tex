\section{Technischer Bericht}

\subsection{Einleitung und Übersicht}
% Hier soll beschrieben werden wie der technische Bericht aufgebaut ist und was die Kapitel beinhalten.

\subsection{Was ist Xamarin?}
Im Mai 2011 gründete Miguel De Icaza, welcher 2001 das Projekt Mono ins Leben gerufen hatte, die Firma Xamarin mit dem Ziel, die Entwicklung für Cross-Platform-Applikationen für Smartphones zu vereinfachen und zu beschleunigen \cite{XamarinWikipedia}.

Beim Projekt Mono handelt es sich um eine quelloffene Implementation von Microsofts .Net Framework. Die Entwicklung plattformunabhängiger Applikationen ist das Ziel des Projektes \cite{XamarinCCVossel}. Die Entwickler achten bei der Implementation auf die Einhaltung der Standards für die Common Language Infrastrucute (CLI) und Common Language Specification (auch .Net Framework genannt) \cite{XamarinQuora}.

Mono wird stetig weiterentwickelt und ist fester Bestandteil der Xamarin Plattform. Softwareentwickler, welche mit der Xamarin Plattform arbeiten, können Apps für iOS, Android und WindowsPhone in C\# schreiben. Der geschriebene Quellcode kann durchschnittlich zu 75 Prozent für alle Plattformen benutzt werden. Die in C\# geschriebenen Applikationen werden von der Xamarin Plattform in die jeweilige native Sprache übersetzt, was gewährleistet, dass am Ende des Entwicklungsprozesses eine native Applikation für das jeweilige Betriebsystem zur Verfügung steht. Neben dem mobilen Betriebssystemen ist auch die Entwicklung für Mac und Windows möglich \cite{XamarinCCVossel}.

Im Februar 2016 wurde Xamarin von Microsoft aufgekauft und ist seither eine Tochtergesellschaft von Microsoft mit Sitz in San Francisco \cite{XamarinWikipedia}.


\subsection{Was ist die Methode 635?} \label{subsec:methode_635_desc}
Die gesamte Beschreibung der Methode 635 wurde von \href{https://kreativitätstechniken.info/6-3-5-methode/}{kreativitästechniken.info} \cite{methode-635} übernommen. 


Die Methode 635 (auch Methode 6-3-5 geschrieben) ist eine Brainwriting-Kre\-ati\-vi\-täts\-technik. Der Name leitet sich aus den drei wesentlichen Eigenschaften der Methode ab: jeweils 6 Teilnehmer erhalten ein Blatt Papier, auf dem sie je 3 Ideen notieren und die Blätter dann insgesamt 5 mal weiterreichen.

\subsubsection*{Anwendungsgebiete der Methode 635}
Die Methode 635 ist eine Variante des Brainwriting. Sie eignet sich besonders für die erste Phase im kreativen Prozess. Dabei werden zunächst Ideen gesammelt ohne dass eine Bewertung stattfindet. Im Idealfall können so in kurzer Zeit 108 Ideen entstehen. Die Aufforderung, bestehende Ideen aufzugreifen und weiterzuentwickeln macht die Methode 635 zu einer konstruktiven Kreativitätstechnik. Gleichzeitig kann die Kreativität durch die strukturierte Form aber auch gebremst werden. In der Praxis entstehen daher oft etwas weniger Ideen.

\subsubsection*{Vorgehen bei der Methode 635}
Zunächst erklärt der Moderator die Regeln der Methode 635, führt die Teilnehmer in das Ausgangsproblem ein und ist im Folgenden verantwortlich für die Zeitmessung. Sobald die Teilnehmer über die Ausgangsfrage oder -problem aufgeklärt sind, startet der Moderator die erste von sechs Runden. In jeder Runde werden die Teilnehmer aufgerufen, die oberste noch freie Zeile, bestehend aus 3 Kästchen, mit ihren Ideen zu füllen. Dabei sollten die Teilnehmer die Ideen der Vorgänger aufgreifen, erweitern und/oder weiterentwickeln ohne die Ideen zunächst zu bewerten. Nach einer festgelegten Zeit von beispielsweise 5 Minuten beendet der Moderator die Runde. Die Teilnehmer reichen ihr Arbeitsblatt im Uhrzeigersinn an ihren Sitznachbarn weiter und eine neue Runde beginnt. Im Idealfall sind nach 6 Runden genau 6*18 = 108 Ideen entstanden. In Realität ist die Anzahl aufgrund von doppelten oder leeren Einträgen wahrscheinlich etwas geringer. Dennoch sollten nun eine Vielzahl von Ideen vorliegen.


Nun kann eine Diskussion, Analyse und Bewertung der Ideen erfolgen.

\newpage

\subsection{Vorstudie}

\subsubsection{Xamarin.Forms oder Xamarin native}

\subsubsection{Visual Studio App Center}

\subsubsection{Backend-Technologie}

\subsubsection{Methode 635 als verteiltes System}
%TODO erwähnen, dass wir über ein verteiltes System nachgedacht haben? 
%Nach kurzem Gespräch mit Thomas Bocek hat er davon abgeraten. Ein verteiltes System sei immer komplexer und komplizierter als ein Server/Client System. Für diese geringe Anzahl von Teilnehmern, welche prinzipiell nur Messages austauschen, lohnt es sich nicht ein verteiltes System zu bauen. 
\newpage

\subsection{Anforderungsspezifikation}

\subsubsection{Funktionale Anforderungen}

\subsubsection{Nicht-Funktionale Anforderungen}
Beim Thema Nicht-Funktionale Anforderungen halten wir uns an die Standards ISO 9126\cite{ISO9126} bzw. dessen Nachfolger ISO 25010\cite{ISO9126_ISO25010}. Beide ISO-Normen sind sich sehr ähnlich und liefern eine gute Checkliste für jegliche Art von Systemanforderungen.

\begin{figure}[h]
	\centering
	\includegraphics[width=1\linewidth]{img/anforderungen/quality}
	\caption[Anforderungskategorien nach ISO 25010]{Anforderungskategorien nach  ISO 25010}
	\label{fig:ISO 25010}
\end{figure}

%TODO Bild Link: https://blog.seibert-media.net/blog/2018/05/14/qualitaet-funktionale-und-nichtfunktionale-anforderungen-in-der-software-entwicklung/

Diese Normen sind sehr umfangreich gestaltet. Wir werden uns daher auf die, für uns, wichtigsten Anforderungen konzentrieren.

\begin{description}[leftmargin=!,labelwidth=\widthof{\bfseries Wiederherstellbarkeit}]
	\item[Ressourcennutzung] Die internen Ressourcen wie Kamera, Dateisystem, Batterie usw. müssen sparsam und dürfen nur falls nötig eingesetzt werden.
	\item[Bedienbarkeit] Die Bedienbarkeit der Applikation muss einfach und intuitiv gestaltet sein.
	\item[Fehlervermeidung] Dem Benutzer müssen im Falle eines Fehler einfache und verständliche Fehlermeldungen präsentiert werden.
	\item[Ästhetik] Die Benutzeroberflächen der Applikation dürfen sich nicht wesentlich von den in Xamarin verwendetet Formen unterscheiden.
	\item[Vertraulichkeit] Die Daten der Endbenutzer müssen zu jedem Zeitpunkt vertraulich behandelt werden.
	\item[Anpassbarkeit] Die Anpassung oder Integration von neuen Brainstorming-Methoden muss gewährleistet sein.
	\item[Installierbarkeit] Die Installation der Applikation auf einem Endgerät muss einfach gestaltet sein.
	\item[Co-Existenz] Sollte zu einem späteren Zeitpunkt entschieden werden ein Web-Frontend zu programmieren, muss dieses co-existent mit der Xamarin Applikation existieren können.
	\item[Wiederherstellbarkeit] Im Falle eines fehlerhaften Features, muss es innerhalb eines Werktages möglich sein, die Applikation wieder auf den alten Stand zurück zu holen und erneut zu deployen.	
	\item[Wiederverwendbarkeit] Es ist darauf zu achten, dass der geschriebene Code, falls möglich, wiederverwendet wird.
	\item[Analysierbarkeit] Es muss möglich sein, die Endnutzer und deren Verhalten mit der Applikation zu analysieren.	
\end{description}

\newpage

\input{./tex/techn_bericht/domain_analyse/domain_analyse}
\newpage

\subsection{Architekturdokumentation}

\subsubsection{Logische Architektur}
Wie aus der Abbildung \ref{fig:architektur-methode635} ersichtlich ist, teilen wir die Architektur des gesamten Systems in drei Schichten auf. Die Präsentationsschicht ist die Schicht über die der Benutzer mit der Xamarin App kommuniziert. Diese umfasst im wesentlichen die verschiedenen View-Komponenten. Jegliche Interaktionen über die Oberfläche werden anschliessend in der logischen Schicht weiter verarbeitet. In dieser Schicht haben wir zum einen wieder unsere Xamarin App, welche selbst Logik-Komponenten wie die Timing-Komponente oder weitere App spezifische Logik-Komponenten enthält. 

Zum anderen haben wir das PlayFramework, welches wiederum Access-Komponenten, Routing-Komponenten, eine Timing-Komponente, Logik-Komponenten und eine DataAccess-Komponente enthält.

Mit der Schicht der Datenhaltung (Persistence) haben wir eine Schicht zur Verfügung, welche eine Persistence-Komponente hält.

Konkret steht uns je ein DataStore für die \textit{BrainstormingFindings}, für die \textit{BrainstormingTeams} und für die \textit{Participants} zur Verfügung.


\begin{figure}[h]
	\centering
	\includegraphics[width=1\linewidth]{img/architektur/CD_Methode635}
	\caption{Logische Architektur BrainingOutOfBox}
	\label{fig:architektur-methode635}
\end{figure}

\paragraph*{Komponenten}

\begin{description}[leftmargin=!,labelwidth=\widthof{\bfseries DataAccessComponent}]
	\item[ViewComponent] Die einzelnen View-Komponenten der Xamarin App sind für das korrekte Anzeigen der Informationen verantwortlich. Sie definieren im wesentlichen das Aussehen der Applikation.
	\item[AppTimingComponent] Die Xamarin App hält in der logischen Schicht eine Timing-Komponente, welche dafür sorgt, dass ein \textit{BrainstormingFinding} nach Ablauf der Zeit abgesendet wird.
	\item[AppLogicComponent] Die Logik-Komponente der Xamarin App regelt weitere Logik, wie z.B. den Zugriff auf das PlayFramework.
	\item[AccessComponent] Die Access-Komponente auf dem PlayFramework regelt den Zugriff mittels JWT-Token\cite{jwt}.
	\item[Routing] Die Routing-Komponente sorgt anhand des URLs für das Aufrufen der korrekten Funktion.
	\item[TimingComponent] Wie die Xamarin App hält auch das PlayFramework eine Timing-Komponente, um den Zustand der Zeit halten zu können.
	\item[LogicComponent] In der Logik-Komponente werden die eigentlichen Funktionen geschrieben. Hier ist auch die Logik für den Austausch der Blätter untergebracht.
	\item[DataAccessComponent] Mit DataAccess-Komponente kann auf die Daten zugegriffen werden.
	\item[PersistenceComponent] Die Persistence-Komponente ist für das Speichern der Daten verantwortlich.
\end{description}


\subsubsection{Deployment}
Wie in der Abbildung \ref{fig:deployment-methode635} zu sehen ist, besteht unser System aus zwei physikalischen Geräten. Das ist zum einen der client und zum anderen der backendNode. Diese beinhalten jeweils sogenannte \textit{DeploymentUnits} (DU). 

Beim client handelt es sich im Grunde um das Smartphone des jeweiligen Benutzers. Auf seinem Smartphone läuft dann die Xamarin App, welche wiederum die appPresentationLayerDU und die appBusinessLogicLayerDU hält.

Der backendNode ist ein Ubuntu 18.04 auf dem ein Java Runtime Environment (JRE) installiert ist. Innerhalb der JRE läuft das PlayFramework, in dem wiederum die businessLogicLayerDU läuft.

Zudem ist auf dem backendNode ein MongoDB Service installiert, welche  die persistenceLogicLayerDU beinhaltet.

\begin{figure}[h]
	\centering
	\includegraphics[width=1\linewidth]{img/deployment/DD_Methode635}
	\caption{Deploymentdiagramm BrainingOutOfBox}
	\label{fig:deployment-methode635}
\end{figure}

\paragraph*{Komponenten}

\begin{description}[leftmargin=!,labelwidth=\widthof{\bfseries appBusinessLogicLayerDU}]
	\item[businessLogicLayerDU] Die businessLogicLayerDU enthält alle Komponenten, welche in Abbildung \ref{fig:architektur-methode635} in der logischen Schicht im PlayFramework eingezeichnet sind.
	\item[persistenceLogicLayerDU] Die persistenceLogicLayerDU enthält alle Komponenten, welche in Abbildung \ref{fig:architektur-methode635} in der persistence Schicht eingezeichnet sind.
	\item[appPresentationLayerDU] Die appPresentationLayerDU enthält alle Komponenten, welche in Abbildung \ref{fig:architektur-methode635} in der Präsentationsschicht eingezeichnet sind.
	\item[appBusinessLogicLayerDU] Die appBusinessLogicLayerDU enthält alle Komponenten, welche in Abbildung \ref{fig:architektur-methode635} in der logischen Schicht in der Xamarin App eingezeichnet sind.
\end{description}

\newpage

\subsection{Architekturentscheide}

\subsubsection{Erste Erfahrungen mit der Methode 635}
Wie in Kapitel \ref{subsub:erste_erfahrungen_mit_methode_635} beschrieben, tendiert der Mensch dazu, die Ideen der anderen Teilnehmer automatisch zu bewerten. Als mögliche Lösung wurde die Integration einer Bewertungsmöglichkeit beschrieben.


Wir haben uns allerdings darauf geeinigt, dass wir die originale Version, also ohne die Möglichkeit für eine Wertung, als Vorlage nehmen und diese auch so in unserer Cross-Plattform Applikation umsetzen. 


Die Integration einer Bewertungsmöglichkeit wird als optionales Feature angesehen und lediglich bei genügend Restzeit im Projekt umgesetzt.

\subsubsection{Xamarin.Forms oder Xamarin native}
Für diesen Entscheid galt es zu evaluieren, welche User Controls für unsere Applikation die exotischsten sind. Dies, weil Xamarin.Forms eine Menge an Standard-Controls anbietet, die vom Framework selber in das jeweilige Betriebssystem konvertiert werden. Sind alle vorgesehenen Benutzerelemente in Forms enthalten, sparen wir uns die Zeit, betriebssystemspezifische Elemente zu entwickeln. 

Für unser Projekt haben wir folgende Benutzerelemente als exotisch oder kritisch definiert:
\begin{itemize}
	\item Canvas Control für Zeichnen einer Idee
	\item Camera Funktion für das Erkennen von Quick Response-Codes (QR-Codes)
	\item Verarbeitung und Generierung von QR-Codes
\end{itemize}

Nach einer Recherche stellte sich heraus, dass sich ein Canvas View von Google namens SkiaSharp \cite{skiaSharp} eignet. Darauf lässt sich gemäss der Dokumentation zeichnen sowie definierte Formen einfügen. Dies könnte auch für eine Erweiterung spannend sein, in der Patterns in UML als Vorlage angeboten werden können.

Für die Kamera-Funktionalität steht ein NuGet-Packet (Xam.Media.Plugin \cite{xam-media-plugin}) bereit, das uns diese Arbeit abnehmen wird.

Das Generieren und Lesen der QR-Codes ist an sich kein Problem von Xamarin.Forms, denn grundsätzlich müssen die von der Kamera generierten Files eingelesen und ins entsprechende QR-Code-Tool eingefügt werden. Hierfür eignet sich das NuGet ZXing.Net (\cite{zxing.net}). 

Es stellte sich relativ rasch heraus, dass die gewünschten Funktionalitäten in Xamarin.Forms in ausreichender Qualität enthalten sind und uns das individuelle Entwickeln dadurch abgenommen wird.

\subsubsection{Backend-Technologie}
Neben den Vorteilen, wie der schlanken und zustandslosen Architektur des Play Frameworks, dem asynchronen und nicht-blockierenden Verhalten und den vielen unterstützten Bibliotheken, haben wir uns hauptsächlich dafür entschieden, weil wir in anderen Projekten schon sehr gute Erfahrungen mit dem Play Framework gemacht haben.

Ein weiterer Grund bestand darin, dass das Play Framework nicht nur in Scala sondern auch in Java geschrieben ist. Mit Java kennen wir uns beide gut aus und mussten uns so keine neue Programmiersprache aneignen.

Da wir uns schon relativ früh für eine MongoDB als Datenbanksystem entschieden hatten, viel die Wahl für das Play Framework erst recht, als wir einen asynchronen MongoDB-Treiber für Java gefunden hatten.

\subsubsection{MongoDB als Datenbanksystem}
%TODO sollen wir noch ein Kapitel einfügen und erklären warum wir uns für MongoDB entschieden haben?
%No-SQL DB einfacher skalierbar, hoch verfügbar, CAP & BASE Theorem etc.

\subsubsection{Methode 635 als Peer-to-Peer-System}
Wir haben uns auch überlegt, die Cross-Plattform Applikation d.h. vor allem die Kommunikation zwischen den einzelnen Teilnehmer, als Peer-to-Peer System \cite{Peer2Peer} zu konzipieren.


Prof. Thomas Bocek, Professor für verteilte Systeme an der Hochschule Rapperswil, hat uns allerdings davon abgeraten. Ein verteiltes System sei immer komplexer und komplizierter als ein Server/Client System. Für diese geringe Anzahl von Teilnehmern, welche prinzipiell nur Messages austauschen, lohnt es sich nicht ein verteiltes System zu bauen. 


Daher haben wir uns für eine klassische Server/Client Architektur entschieden.

\subsubsection{Kommunikation zwischen Server und App}
Die Kommunikation zwischen dem zentralen Server und den Clients, also den Cross-Plattform Applikationen in unserem Fall, ist von grosser Bedeutung. Wegen der Problematik der Network Address Translation kurz NAT \cite{NAT} kann diese prinzipiell auf zwei Arten erfolgen: Entweder man verwendet Websockets \cite{WebSockets}, welche eine permanente Verbindung zwischen Server und Client öffnen oder die Kommunikation beginnt ausschliesslich beim Client. 


Wir haben uns für das stetige Abfragen von Informationen (Polling) entschieden, da es eine einfache Variante darstellt. Zwar werden dadurch vermeidbare Requests an den Server gesendet, da aber die Anzahl an Teilnehmer bzw. die Ressourcennutzung des Backends in einem vertretbaren Rahmen liegt, ist diese Polling-Variante völlig in Ordnung.  

\textbf{Zwei Beispiele für Polling in der Applikation}: Wenn ein Teilnehmer seine Ideen aufschreibt, fragt die Applikation im Hintergrund in regelmässigen Abständen den Server nach der verbleibenden Zeit für diese Runde ab.


Auch nach der Abgabe der aufgeschriebenen Ideen an den Server, muss der Teilnehmer warten, bis er die Ideen bzw. das Blatt seines Nachbarn bekommt. Dafür muss die Applikation immer wieder den Server fragen, ob der Nachbar überhaupt sein Blatt bzw. seine Ideen abgegeben hat. Denn vorher kann der Teilnehmer auch nicht weitermachen.


Andere Varianten der Kommunikation, wie Webhooks oder Websockets werden daher nicht weiter verfolgt.
\newpage

\subsection{Herausforderungen}
Hier sind besonders erwähnenswerte Herausforderungen und Hürden beschrieben, die sich im Verlaufe des Projektes gestellt haben. Dies soll anderen Software Ingenieuren oder Interessierten helfen  aus unseren Schwierigkeiten zu lernen. 

\subsubsection{HTTPS REST Schnittstelle in CI/CD aufsetzen}
Während dem Entwickeln des Prototyps in der Evaluation stellten wir fest, dass das Abfragen unserer Backend-Schnittstelle auf Android wie gewünscht funktionierte, jedoch warf die Applikation beim Ausführen auf iOS eine Exception. Der Grund dafür war, dass iOS keine Verbindungen zu unverschlüsselten Webseiten mehr zulässt. Daher war der Fall klar, dass dies noch aufgesetzt werden muss.

Nach kurzen Recherchen haben wir für das Erstellen und Validieren des Zertifikates auf Let's Encrypt \cite{letsencrypt} gewählt, gerade deshalb weil eine ausführliche Dokumentation und eine rasche Generierung möglich ist. 

Auch mit dem verwendeten Play Framework sollte das Aufsetzen der HTTPS Site kein Problem darstellen, es sollte mit einem Parameter beim Start der Applikation gut möglich sein. 

Nachdem das Zertifikat installiert wurde und die Applikation lokal erfolgreich lief, galt es, das Gesamte noch in den Build-Prozess einzubauen. Dabei kam das Problem auf, dass der Pfad zum Keystore, der das Let's Encrypt-Zertifikat beinhaltet, nicht gültig war. Nach mehrmaligem, gründlichem Überprüfen des Pfades und neu Builden, wurden immer noch Exceptions geworfen.

Wir haben keinen Anhaltspunkt, was der Fehler sein könnte, denn werden die genau gleichen Befehle auf dem Backend-Server direkt ausgeführt, funktioniert die Applikation einwandfrei auf HTTPS. Sobald es aber über den Build-Server läuft, findet er den Pfad zum Keystore nicht mehr.

Als Work-Around haben wir eine \texttt{CustomSslEngineProvider} geschrieben, der den Pfad zum Keystore direkt im Code beinhaltet. Darauf hat Puppet keinen Einfluss und die Applikation funktioniert wie gewünscht.

%TODO Herausforderung mit angedachtem Layout beschreiben. War so nicht umsetzbar.
\newpage

\subsection{Ergebnisse}
Das Kapitel der Ergebnisse befasst sich mit der konkreten Umsetzung der Xamarin Applikation und dem PlayFramework. Dabei werden die wichtigsten Funktionen der jeweiligen Bestandteile genauer erklärt und beim Xamarin App zusätzlich noch eine kurze Retrospektive auf geplantes Design und effektiver Umsetzung gegeben.

Zur besseren Übersicht wurde der Code vereinzelt gekürzt. Dies ist durch 3 Punkte (...) gekennzeichnet.

\subsubsection{Implementierung des PlayFrameworks}
Die Umsetzung des Backends basiert auf dem PlayFramework. Die Gründe für diesen Entscheid kann in Kapitel \ref{vorstudie} nachgelesen werden. 

Um die Daten permanent zu speichern, besitzt das PlayFramework eine Anbindung an eine MongoDB Instanz. Die Anbindung wurde mit dem MongoDB Async Driver für Java \cite{MongoDBAsyncDriver} realisiert. Die Implementation ist daher auch in Java geschrieben. Für die Dokumentation des Backends haben wir Swagger \cite{swagger} verwendet.

\paragraph*{ParticipantController.java}

\lstset{language=JAVA, showstringspaces=false, frame=single, captionpos=b, label=createParticipant, breaklines=true, numbers=left}
\begin{lstlisting}[caption={Participant erstellen}]
public Result createParticipant(){

    JsonNode body = request().body().asJson();

    if (body == null ) {
        return forbidden(Json.toJson(new ErrorMessage("Error", "json body is null")));
    } else if(  body.hasNonNull("username") &&
            	body.hasNonNull("password") &&
            	body.hasNonNull("firstname") &&
            	body.hasNonNull("lastname")) {

        Participant participant = new Participant(body.get("username").asText(), body.get("password").asText(), body.get("firstname").asText(), body.get("lastname").asText());

        participantCollection.insertOne(participant, new SingleResultCallback<Void>() {
            @Override
            public void onResult(Void result, Throwable t) {
                Logger.info("Inserted Participant!");
            }
        });

        return ok(Json.toJson(new SuccessMessage("Success", "Participant successfully inserted")));
    }

    return forbidden(Json.toJson(new ErrorMessage("Error", "json body not as expected")));
}
\end{lstlisting}

Bei der Methode für das Erstellen von Participants, wird zuerst geprüft, ob ein HTML-Body existiert. Ist dies nicht der Fall, wird ein HTTP-Response Status-Code (Zeile 24) zurückgesendet.

Existiert ein HTML-Body, wird dieser auf die Existenz der Felder \textit{username}, \textit{password}, \textit{firstname} und \textit{lastname} geprüft (Zeile 7-10). Daraus wird als nächstes ein \texttt{participant} erstellt (Zeile 12) und mittels  \texttt{insertOne} in die \texttt{participant\-Collection} gespeichert (Zeile 14).

Zulezt wird der Status-Code 200 mit einer Nachricht an den Absender zurück gesendet.

\begin{lstlisting}[caption={Login}]
public Result login() throws UnsupportedEncodingException, ExecutionException, InterruptedException {
...
if (body.hasNonNull("username") && body.hasNonNull("password")) {
    CompletableFuture<Participant> future = new CompletableFuture<>();

    participantCollection.find(and(
    eq("username", body.get("username").asText()),
    eq("password", body.get("password").asText()))).first(new SingleResultCallback<Participant>() {
        @Override
        public void onResult(Participant participant, Throwable t) {
            if (participant != null) {
                Logger.info("Found participant");
                future.complete(participant);
            } else {
                future.complete(null);
            }
        }
    });

    if (future.get()!= null){
        ObjectNode result = Json.newObject();
        result.putPOJO("participant", future.get());
        result.put("access_token", getSignedToken(7l));
        return ok(result);
    } else {
       ...
    }

} else {
    ...
}

}
\end{lstlisting}

Für das Login eines Participants wird auch hier zuerst nach der Existenz der Felder \textit{username} und \textit{password} im HTML-Body geschaut (Zeile 3). Im nächsten Schritt wird die Datenbank nach einem Participant mit den angegebenen Werten durchsucht (Zeile 6-8). 

Da wir einen asynchronen Treiber verwenden, benötigen wir ein \texttt{CompletableFuture}, um das Resultat der Abfrage darin zu speichern und um mittels \texttt{future.get()} darauf zugreifen zu können (Zeile 20).

Am Ende wird neben dem gefundenen \textit{participant} noch ein \textit{JWT-Token} dem Resultat angefügt (Zeile 21-24).

\paragraph*{TeamController.java}
\begin{lstlisting}[caption={Einem Team beitreten}]
public Result joinBrainstormingTeam(String teamIdentifier) throws ExecutionException, InterruptedException {
...
if (brainstormingTeam!= null && brainstormingTeam.getNrOfParticipants() > brainstormingTeam.getCurrentNrOfParticipants() && brainstormingTeam.joinTeam(participant)) {

    teamCollection.updateOne(eq("identifier", teamIdentifier),combine(set("participants", brainstormingTeam.getParticipants()), inc("currentNrOfParticipants", 1)), new SingleResultCallback<UpdateResult>() {
        @Override
        public void onResult(final UpdateResult result, final Throwable t) {
            Logger.info(result.getModifiedCount() + " Team successfully updated");
        }
    });

    return ok(Json.toJson(new SuccessMessage("Success", "Participant successfully added to the brainstormingTeam")));

} else {
    ...
}
...
}
\end{lstlisting}

Dieses Beispiel soll zeigen, wie mit dem MongoDB Async Driver ein Eintrag mittels \texttt{updateOne} aktualisiert werden kann.

Wie auch schon bei der \texttt{find} Methode, kennzeichnet der erste Parameter das Dokument, welches man aktualisieren möchte. Der zweite Parameter steht für die Felder und deren neuen Werte und der dritte und letzte Parameter ist wieder der \texttt{SingleResultCallback} und beschreibt wie das Resultat weiter prozessiert wird. All dies ist auf Zeile 5-8 zu finden.

\paragraph*{FindingController.java}
\begin{lstlisting}[caption={Watcher für BrainstormingFinding}]
private void startWatcherForBrainstormingFinding(String identifier){

ScheduledExecutorService executor = Executors.newSingleThreadScheduledExecutor();

TimerTask task = new TimerTask() {
    @Override
    public void run() {
        try {
    	...
	if (finding.getCurrentRound() > finding.getBrainsheets().size()){
	    lastRound(identifier);
	    executor.shutdown();
	}

	if (endDateTime.plusSeconds(30).isBeforeNow() ||
	finding.getDeliveredBrainsheetsInCurrentRound() >= finding.getBrainsheets().size()){
	nextRound(identifier);
	}
            cancel();

        } catch (ExecutionException e) {
            e.printStackTrace();
        } catch (InterruptedException e) {
            e.printStackTrace();
        }
    }
};

executor.scheduleAtFixedRate(task, 1000L, 5000L, TimeUnit.MILLISECONDS);
}
\end{lstlisting}

Um den Zustand über die verbleibende Zeit oder die bereits eingereichten \textit{Brainsheets} überwachen zu können, setzen wir einen \texttt{ScheduledExecutorService} \cite{JavaTimer} ein. Dieser erlaubt uns alle 5000ms bzw. alle 5s den \texttt{TimerTask} auf Zeile 5 auszuführen. 

Der \texttt{TimerTask} prüft zuerst, ob die aktuelle Runde schon die letzte Runde ist (Zeile 10). Ist dies der Fall, führt er die Methode \textit{lastRound(identifier)} aus und beendet den executor, sodass keine neuen \texttt{TimerTask} Objekte gestartet werden. In diesem Zustand ist das gesamte \textit{BrainstormingFinding} ausgefüllt.

Ist dies nicht der Fall, prüft er als nächstes, ob die Endzeit der aktuellen Runde plus 30s noch vor der aktuellen Uhrzeit liegt (Zeile 15). Die Bedingung auf Zeile 16 prüft, ob alle \textit{Brainsheets}, welche für die Abgabe erwartet werden schon abgegeben wurden. Unabhängig welche dieser zwei Bedingungen (Zeile 15 oder 16) zuerst eintrifft, es wird anschliessend immer die Methode \textit{nextRound(identifier)} ausgeführt.

Sollte keine der Bedingungen (Zeile 10, 15 oder 16) zutreffen, so beendet sich der \texttt{TimerTask} mittels \textit{cancel} selbst. 

Da der executor aber nach 5s den nächsten \texttt{TimerTask} startet, ist so für die gesamte Dauer, für die das \textit{BrainstormingFinding} läuft, ein 'Watcher' für den korrekten Ablauf zuständig. Die Methode \textit{startWatcherForBrainstormingFinding(String identifier)} wird beim Start eines \textit{BrainstormingFinding} ausgeführt.

\begin{lstlisting}
public Result startBrainstorming(String findingIdentifer) throws ExecutionException, InterruptedException {
        startWatcherForBrainstormingFinding(findingIdentifer);
        return nextRound(findingIdentifer);
}
\end{lstlisting}
\subsubsection{Implementierung der Xamarin App}
\newpage


\subsection{Schlussfolgerungen}
\subsubsection{Ergebnisbewertung}
\subsubsection{Ausblick}