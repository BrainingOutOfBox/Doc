\section{Technischer Bericht}

\subsection{Einleitung und Übersicht}

\subsection{Was ist Xamarin?}


\subsection{Was ist die Methode 635?} \label{subsec:methode_635_desc}
Die gesamte Beschreibung der Methode 635 wurde von \href{https://kreativitätstechniken.info/6-3-5-methode/}{kreativitästechniken.info} \cite{methode-635} übernommen. 


Die Methode 635 (auch Methode 6-3-5 geschrieben) ist eine Brainwriting-Kre\-ati\-vi\-täts\-technik. Der Name leitet sich aus den drei wesentlichen Eigenschaften der Methode ab: jeweils 6 Teilnehmer erhalten ein Blatt Papier, auf dem sie je 3 Ideen notieren und die Blätter dann insgesamt 5 mal weiterreichen.

\subsubsection*{Anwendungsgebiete der Methode 635}
Die Methode 635 ist eine Variante des Brainwriting. Sie eignet sich besonders für die erste Phase im kreativen Prozess. Dabei werden zunächst Ideen gesammelt ohne dass eine Bewertung stattfindet. Im Idealfall können so in kurzer Zeit 108 Ideen entstehen. Die Aufforderung, bestehende Ideen aufzugreifen und weiterzuentwickeln macht die Methode 635 zu einer konstruktiven Kreativitätstechnik. Gleichzeitig kann die Kreativität durch die strukturierte Form aber auch gebremst werden. In der Praxis entstehen daher oft etwas weniger Ideen.

\subsubsection*{Vorgehen bei der Methode 635}
Zunächst erklärt der Moderator die Regeln der Methode 635, führt die Teilnehmer in das Ausgangsproblem ein und ist im Folgenden verantwortlich für die Zeitmessung. Sobald die Teilnehmer über die Ausgangsfrage oder -problem aufgeklärt sind, startet der Moderator die erste von sechs Runden. In jeder Runde werden die Teilnehmer aufgerufen, die oberste noch freie Zeile, bestehend aus 3 Kästchen, mit ihren Ideen zu füllen. Dabei sollten die Teilnehmer die Ideen der Vorgänger aufgreifen, erweitern und/oder weiterentwickeln ohne die Ideen zunächst zu bewerten. Nach einer festgelegten Zeit von beispielsweise 5 Minuten beendet der Moderator die Runde. Die Teilnehmer reichen ihr Arbeitsblatt im Uhrzeigersinn an ihren Sitznachbarn weiter und eine neue Runde beginnt. Im Idealfall sind nach 6 Runden genau 6*18 = 108 Ideen entstanden. In Realität ist die Anzahl aufgrund von doppelten oder leeren Einträgen wahrscheinlich etwas geringer. Dennoch sollten nun eine Vielzahl von Ideen vorliegen.


Nun kann eine Diskussion, Analyse und Bewertung der Ideen erfolgen.

\newpage

% Hier soll beschrieben werden wie der technische Bericht aufgebaut ist und was die Kapitel beinhalten.

\subsection{Vorstudie}

\subsubsection{Xamarin.Forms oder Xamarin native}

\subsubsection{Visual Studio App Center}

\subsubsection{Backend-Technologie}

\subsubsection{Methode 635 als verteiltes System}
%TODO erwähnen, dass wir über ein verteiltes System nachgedacht haben? 
%Nach kurzem Gespräch mit Thomas Bocek hat er davon abgeraten. Ein verteiltes System sei immer komplexer und komplizierter als ein Server/Client System. Für diese geringe Anzahl von Teilnehmern, welche prinzipiell nur Messages austauschen, lohnt es sich nicht ein verteiltes System zu bauen. 
\newpage

\subsection{Anforderungsspezifikation}

\subsubsection{Funktionale Anforderungen}

\subsubsection{Nicht-Funktionale Anforderungen}
Beim Thema Nicht-Funktionale Anforderungen halten wir uns an die Standards ISO 9126\cite{ISO9126} bzw. dessen Nachfolger ISO 25010\cite{ISO9126_ISO25010}. Beide ISO-Normen sind sich sehr ähnlich und liefern eine gute Checkliste für jegliche Art von Systemanforderungen.

\begin{figure}[h]
	\centering
	\includegraphics[width=1\linewidth]{img/anforderungen/quality}
	\caption[Anforderungskategorien nach ISO 25010]{Anforderungskategorien nach  ISO 25010}
	\label{fig:ISO 25010}
\end{figure}

%TODO Bild Link: https://blog.seibert-media.net/blog/2018/05/14/qualitaet-funktionale-und-nichtfunktionale-anforderungen-in-der-software-entwicklung/

Diese Normen sind sehr umfangreich gestaltet. Wir werden uns daher auf die, für uns, wichtigsten Anforderungen konzentrieren.

\begin{description}[leftmargin=!,labelwidth=\widthof{\bfseries Wiederherstellbarkeit}]
	\item[Ressourcennutzung] Die internen Ressourcen wie Kamera, Dateisystem, Batterie usw. müssen sparsam und dürfen nur falls nötig eingesetzt werden.
	\item[Bedienbarkeit] Die Bedienbarkeit der Applikation muss einfach und intuitiv gestaltet sein.
	\item[Fehlervermeidung] Dem Benutzer müssen im Falle eines Fehler einfache und verständliche Fehlermeldungen präsentiert werden.
	\item[Ästhetik] Die Benutzeroberflächen der Applikation dürfen sich nicht wesentlich von den in Xamarin verwendetet Formen unterscheiden.
	\item[Vertraulichkeit] Die Daten der Endbenutzer müssen zu jedem Zeitpunkt vertraulich behandelt werden.
	\item[Anpassbarkeit] Die Anpassung oder Integration von neuen Brainstorming-Methoden muss gewährleistet sein.
	\item[Installierbarkeit] Die Installation der Applikation auf einem Endgerät muss einfach gestaltet sein.
	\item[Co-Existenz] Sollte zu einem späteren Zeitpunkt entschieden werden ein Web-Frontend zu programmieren, muss dieses co-existent mit der Xamarin Applikation existieren können.
	\item[Wiederherstellbarkeit] Im Falle eines fehlerhaften Features, muss es innerhalb eines Werktages möglich sein, die Applikation wieder auf den alten Stand zurück zu holen und erneut zu deployen.	
	\item[Wiederverwendbarkeit] Es ist darauf zu achten, dass der geschriebene Code, falls möglich, wiederverwendet wird.
	\item[Analysierbarkeit] Es muss möglich sein, die Endnutzer und deren Verhalten mit der Applikation zu analysieren.	
\end{description}

\newpage

\input{./tex/techn_bericht/domain_analyse/domain_analyse}
\newpage

\subsection{Architekturentscheide}

\subsubsection{Erste Erfahrungen mit der Methode 635}
Wie in Kapitel \ref{subsub:erste_erfahrungen_mit_methode_635} beschrieben, tendiert der Mensch dazu, die Ideen der anderen Teilnehmer automatisch zu bewerten. Als mögliche Lösung wurde die Integration einer Bewertungsmöglichkeit beschrieben.


Wir haben uns allerdings darauf geeinigt, dass wir die originale Version, also ohne die Möglichkeit für eine Wertung, als Vorlage nehmen und diese auch so in unserer Cross-Plattform Applikation umsetzen. 


Die Integration einer Bewertungsmöglichkeit wird als optionales Feature angesehen und lediglich bei genügend Restzeit im Projekt umgesetzt.

\subsubsection{Xamarin.Forms oder Xamarin native}
Für diesen Entscheid galt es zu evaluieren, welche User Controls für unsere Applikation die exotischsten sind. Dies, weil Xamarin.Forms eine Menge an Standard-Controls anbietet, die vom Framework selber in das jeweilige Betriebssystem konvertiert werden. Sind alle vorgesehenen Benutzerelemente in Forms enthalten, sparen wir uns die Zeit, betriebssystemspezifische Elemente zu entwickeln. 

Für unser Projekt haben wir folgende Benutzerelemente als exotisch oder kritisch definiert:
\begin{itemize}
	\item Canvas Control für Zeichnen einer Idee
	\item Camera Funktion für das Erkennen von Quick Response-Codes (QR-Codes)
	\item Verarbeitung und Generierung von QR-Codes
\end{itemize}

Nach einer Recherche stellte sich heraus, dass sich ein Canvas View von Google namens SkiaSharp \cite{skiaSharp} eignet. Darauf lässt sich gemäss der Dokumentation zeichnen sowie definierte Formen einfügen. Dies könnte auch für eine Erweiterung spannend sein, in der Patterns in UML als Vorlage angeboten werden können.

Für die Kamera-Funktionalität steht ein NuGet-Packet (Xam.Media.Plugin \cite{xam-media-plugin}) bereit, das uns diese Arbeit abnehmen wird.

Das Generieren und Lesen der QR-Codes ist an sich kein Problem von Xamarin.Forms, denn grundsätzlich müssen die von der Kamera generierten Files eingelesen und ins entsprechende QR-Code-Tool eingefügt werden. Hierfür eignet sich das NuGet ZXing.Net (\cite{zxing.net}). 

Es stellte sich relativ rasch heraus, dass die gewünschten Funktionalitäten in Xamarin.Forms in ausreichender Qualität enthalten sind und uns das individuelle Entwickeln dadurch abgenommen wird.

\subsubsection{Backend-Technologie}
Neben den Vorteilen, wie der schlanken und zustandslosen Architektur des Play Frameworks, dem asynchronen und nicht-blockierenden Verhalten und den vielen unterstützten Bibliotheken, haben wir uns hauptsächlich dafür entschieden, weil wir in anderen Projekten schon sehr gute Erfahrungen mit dem Play Framework gemacht haben.

Ein weiterer Grund bestand darin, dass das Play Framework nicht nur in Scala sondern auch in Java geschrieben ist. Mit Java kennen wir uns beide gut aus und mussten uns so keine neue Programmiersprache aneignen.

Da wir uns schon relativ früh für eine MongoDB als Datenbanksystem entschieden hatten, viel die Wahl für das Play Framework erst recht, als wir einen asynchronen MongoDB-Treiber für Java gefunden hatten.

\subsubsection{MongoDB als Datenbanksystem}
%TODO sollen wir noch ein Kapitel einfügen und erklären warum wir uns für MongoDB entschieden haben?
%No-SQL DB einfacher skalierbar, hoch verfügbar, CAP & BASE Theorem etc.

\subsubsection{Methode 635 als Peer-to-Peer-System}
Wir haben uns auch überlegt, die Cross-Plattform Applikation d.h. vor allem die Kommunikation zwischen den einzelnen Teilnehmer, als Peer-to-Peer System \cite{Peer2Peer} zu konzipieren.


Prof. Thomas Bocek, Professor für verteilte Systeme an der Hochschule Rapperswil, hat uns allerdings davon abgeraten. Ein verteiltes System sei immer komplexer und komplizierter als ein Server/Client System. Für diese geringe Anzahl von Teilnehmern, welche prinzipiell nur Messages austauschen, lohnt es sich nicht ein verteiltes System zu bauen. 


Daher haben wir uns für eine klassische Server/Client Architektur entschieden.

\subsubsection{Kommunikation zwischen Server und App}
Die Kommunikation zwischen dem zentralen Server und den Clients, also den Cross-Plattform Applikationen in unserem Fall, ist von grosser Bedeutung. Wegen der Problematik der Network Address Translation kurz NAT \cite{NAT} kann diese prinzipiell auf zwei Arten erfolgen: Entweder man verwendet Websockets \cite{WebSockets}, welche eine permanente Verbindung zwischen Server und Client öffnen oder die Kommunikation beginnt ausschliesslich beim Client. 


Wir haben uns für das stetige Abfragen von Informationen (Polling) entschieden, da es eine einfache Variante darstellt. Zwar werden dadurch vermeidbare Requests an den Server gesendet, da aber die Anzahl an Teilnehmer bzw. die Ressourcennutzung des Backends in einem vertretbaren Rahmen liegt, ist diese Polling-Variante völlig in Ordnung.  

\textbf{Zwei Beispiele für Polling in der Applikation}: Wenn ein Teilnehmer seine Ideen aufschreibt, fragt die Applikation im Hintergrund in regelmässigen Abständen den Server nach der verbleibenden Zeit für diese Runde ab.


Auch nach der Abgabe der aufgeschriebenen Ideen an den Server, muss der Teilnehmer warten, bis er die Ideen bzw. das Blatt seines Nachbarn bekommt. Dafür muss die Applikation immer wieder den Server fragen, ob der Nachbar überhaupt sein Blatt bzw. seine Ideen abgegeben hat. Denn vorher kann der Teilnehmer auch nicht weitermachen.


Andere Varianten der Kommunikation, wie Webhooks oder Websockets werden daher nicht weiter verfolgt.
\newpage


\subsection{Ergebnisse}

\subsection{Schlussfolgerungen}
\subsubsection{Ergebnisbewertung}
\subsubsection{Ausblick}