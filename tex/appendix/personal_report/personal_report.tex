\section{Persönliche Berichte}
Um über den Projektverlauf zu reflektieren, schreiben in diesem Kapitel beide Teilnehmer ihre eigenen Erfahrungen auf und beschreiben Besonderheiten, die ihnen während dem Projekt begegnet sind. 

\subsection{Oliver Dias}
Aus meiner Sicht haben wir ein sehr erfolgreiches Projekt hinter uns. Was mich besonders freut, ist, wie wir gerade in der letzten Phase der Entwicklung mit schwierigeren Situationen umgegangen sind. Dabei ging es unter anderem darum abzuwägen, welche Features in der verbleibenden Zeit für ein anwendbares Produkt noch möglich waren. Wir haben uns für eine effiziente Zwischenlösung entschieden und zusätzlich ein Konzept für eine ausführlichere Lösung präsentiert. 

Ich bin überzeugt, dass ich in der Praxis immer wieder an ähnliche Situationen herankommen werde und dann aufgrund von solchen Erfahrungen besser abschätzen kann, was der richtige Entscheid ist. 

Für mich weiter wesentlich für den Projekterfolg ist die gute Kommunikation unter den Projektpartnern und Betreuer. Dies verhalf enorm für das Finden von Lösungsansätzen und das Beseitigen von Unklarheiten.

Das Arbeiten mit Xamarin.Forms beurteile ich als in Ordnung. Einige Male kam ich an Situationen heran, an der ich doch plattform-spezifische Codestücke schreiben musste. Und dennoch funktionierte nicht alles auf beiden Plattformen wie gewünscht. Es gab zum Beispiel Open-Source Controls, bei denen bestimmte Properties nur auf Android funktionierten und iOS sogar zum Absturz brachten. Auch wenn mit Xamarin einen grossen Schritt gemacht wurde, besteht trotzdem noch Potenzial. Aber auch diese Erfahrung schätze ich und bin dennoch froh für die Arbeit, die in die Entwicklung von Xamarin.Forms investiert wurde. 

Rückblickend kann ich sagen, dass ich einen ähnlichen Pfad einschlagen würde. Allerdings gäbe ich besonders in der Entwicklung noch etwas mehr Acht im Bezug auf das Design. Es lohnt sich, einen Moment zurückzulehnen und sich einen Ansatz nochmals in Ruhe zu überlegen, bevor man sich in das Coding-Adventure stürzt.
\newpage

\subsection{Elias Brunner}
Da ich schon vor dem Beginn der Studienarbeit die Idee für eine Smartphone Applikation für die Methode 635 als Projektthema hatte, galt es daher einen Projektpartner zu finden, welcher die Idee auch gut fand. 

Oliver und ich kannten uns zwar schon aus dem Modul Business und Recht 1, hatten bis dato aber noch nicht miteinander an einem Projekt wie diesem zusammengearbeitet. Die Zusammenarbeit mit ihm empfand ich als sehr angenehm, zumal wir eine sehr ähnliche Arbeitshaltung haben. Auch die Kommunikation mit ihm, sowie mit unserem Betreuer Prof. Dr. Zimmermann, war ein wesentlicher Faktor für den Projekterfolg.

Da wir die Arbeiten nach unseren Stärken so aufgeteilt hatten, dass Oliver sich eher mit der Xamarin App auseinandersetzte und ich mich auf das Backend fokussierte, kam ich nur begrenzt in Kontakt mit Xamarin als Cross-Plattform Technologie. Diesen Umstand würde ich, falls möglich bei zukünftigen Arbeiten gerne ändern. Was ich allerdings diesbezüglich aus diesem Projekt als Erfahrung mitnehme ist der Umstand, dass Xamarin in der Theorie zwar viel verspricht, in der Praxis allerdings immer noch Potenzial für Verbesserung aufweist. Gerade während dem Testen auf unseren Smartphones (iOS und Android) kam es immer wieder vor, dass unsere Applikation auf Android lief, auf iOS aber abstürzte. Dies machte die Fehlersuche ziemlich schwierig und aufwendig.

Dennoch empfinde ich den Entscheid Xamarin für unser Projekt zu Verwenden als richtig. Ich bin überzeugt, dass es genau solche Erfahrungen sind, welche mich in der Zeit nach der HSR besser abschätzen lassen, ob dieser oder jener Entscheid besser für ein Projekt ist. 

Wenn ich nun auf die 14 Projektwochen zurück schaue, bin ich schon etwas stolz darauf, was wir in dieser doch sehr kurzen Zeit erreicht haben. Klar gibt es auch noch einige Punkt, welche für eine \grqq produktive\grqq{} Applikation  verbessert werden müssten. Alles in Allem bin ich aber zufrieden mit unserer Arbeit.


