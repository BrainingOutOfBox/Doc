\section{Persönliche Berichte}
Um über den Projektverlauf zu reflektieren, schreiben in diesem Kapitel beide Teilnehmer ihre eigenen Erfahrungen auf und beschreiben Besonderheiten, die ihnen während dem Projekt begegnet sind. 

\subsection{Oliver Dias}
Aus meiner Sicht haben wir ein sehr erfolgreiches Projekt hinter uns. Was mich besonders freut, ist, wie wir gerade in der letzten Phase der Entwicklung mit schwierigeren Situationen umgegangen sind. Dabei ging es unter anderem darum abzuwägen, welche Features in der verbleibenden Zeit für ein anwendbares Produkt noch möglich waren. Wir haben uns für eine effiziente Zwischenlösung entschieden und zusätzlich ein Konzept für eine ausführlichere Lösung präsentiert. 

Ich bin überzeugt, dass ich gerade in der Praxis immer wieder an ähnliche Situationen herankommen werde und dann aufgrund von solchen Erfahrungen besser abschätzen kann, was der richtige Entscheid ist. 

Für mich weiter wesentlich für den Projekterfolg ist die gute Kommunikation unter den Projektpartnern und Betreuer. Dies verhalf enorm für das Finden von Lösungsansätzen und das Beseitigen von Unklarheiten.

Das Arbeiten mit Xamarin Forms beurteile ich als in Ordnung. Einige Male kam ich an Situationen heran, an der ich doch plattform-spezifische Codestücke schreiben musste. Und dennoch funktionierte nicht alles auf beiden Plattformen wie gewünscht. Es gab zum Beispiel Open-Source Controls, bei denen bestimmte Properties nur auf Android funktionierten und iOS sogar zum Absturz brachten. Auch wenn mit Xamarin einen grossen Schritt gemacht wurde, besteht trotzdem noch Potential. Aber auch diese Erfahrung schätze ich und bin dennoch froh für die Arbeit, die in die Entwicklung von Xamarin Forms investiert wurde. 

Rückblickend kann ich sagen, dass ich einen ähnlichen Pfad einschlagen würde. Allerdings gäbe ich besonders in der Entwicklung noch etwas mehr Acht im Bezug auf das Design. Es lohnt sich, einen Moment zurückzulehnen und sich einen Ansatz nochmals in Ruhe zu überlegen, bevor man sich in das Coding-Adventure stürzt..

\subsection{Elias Brunner}