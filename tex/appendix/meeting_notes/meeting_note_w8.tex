\hypertarget{meeting-notes-8.11.2018}{%
\section*{Meeting notes 8.11.2018}\label{meeting-notes-8.11.2018}}

\hypertarget{anwesende}{%
\subsection*{Anwesende}\label{anwesende}}

\begin{itemize}

\item
  Prof. Dr. Olaf Zimmermann
\item
  Elias Brunner
\item
  Oliver Dias
\end{itemize}

\hypertarget{agenda}{%
\subsection*{Agenda}\label{agenda}}

\hypertarget{catch-upux2fstatus}{%
\subsubsection*{Catch-up/Status}\label{catch-upux2fstatus}}

\begin{itemize}

\item
  Zeigen, was alles im Front- bzw. Backend gemacht wurde
\end{itemize}

\hypertarget{review-dokumentation-risiko-mit-https}{%
\subsubsection*{Review Dokumentation Risiko mit
HTTPS}\label{review-dokumentation-risiko-mit-https}}

\begin{itemize}
\item
  Im Unterkapitel 'Herausforderungen'
\item
  Einleitung beinhaltet ein Satz, der nicht vollständig ist.
\item
  Aufwand in Zeit aufzeigen ?
\item
  Inhaltlich gut, sprachlich noch Luft nach oben
\end{itemize}

\hypertarget{review-dokumentation-dd-und-cd}{%
\subsubsection*{Review Dokumentation DD und
CD}\label{review-dokumentation-dd-und-cd}}

\begin{itemize}
\item
  Im Kapitel 4.7 'Architekturdokumentation'
\item
  Inhalt auf gutem Weg
\item
  Hauptsächlich organisatorisches, grammatikalisches
\item
  "wesentlich" vermeiden
\item
  in der Kernterminologie nicht variieren, in der sprachlichen
  Formulierung jedoch schon (nur Verben)
\item
  "haben wir" durch "besteht aus", "weist auf", "positioniert",
  "enthalten" ersetzen
\item
  Darstellung in Grafiken: Frontend links, Backend links, Presentation
  oben, Daten unten
\item
  Leser durch Wiederhohlung über das Projekt lehren, zB mit Referenzen
  auf UseCase1 oder BrainstormingView etc.
\item
  Tautologie: Definition sagt das gleiche wie Begriff nochmals aus.
  Vermeiden!
\item
  physikalisches Gerät -\textgreater{} physikalische Komponente
\item
  exotisch definieren -\textgreater{} wichtig für Projekterfolg
\item
  Glossar momentan nicht nötig
\item
  enthalten reicht nicht, sondern evtl auf NFA Bezug nehmen
\end{itemize}

\hypertarget{projektmanagement}{%
\subsubsection*{Projektmanagement}\label{projektmanagement}}

\begin{itemize}
\item
  Code Freeze um eine Woche verschoben (9.12)
\item
  Generelles Zeitmanagement: für einige Features/NFA reichts wohl nicht
  (Canvas, Usability)
\item
  Grundsatz, lieber Core richtig als "coole Nebenfeatures" halbherzig
\item
  In Doku folgenderamssen argumenterien:

  \begin{itemize}
  
  \item
    Letztes oder Vorletzes Kapitel: Anforderungen und Erreichtes, evtl
    tabellarische Darstellung von NFA
  \item
    Argumentation begränzte Ressourcen
  \item
    Ausblick: Weiterführende Arbeiten/Folgearbeiten
  \item
    Bewusst entscheiden, im Soll-Ist Vergleich gut begründen
  \end{itemize}
\item
  Folgearbeiten von SA auf BA sind möglich
\item
  "Solution Strategy", etwas breiter als nur Patterns, "Design
  Thinking", 635 im Kontext von
\item
  arc42
\item
  Patternsprache: Silvan Gehrig Writers Workshop (mehr Review), Sachen
  verlinkt im aktuellen Wiki Übung von letzter Woche (Woche 7),
  Patternsprache heisst MAP (Microservices API Patterns)
\item
  \begin{enumerate}
  \def\labelenumi{\arabic{enumi}.}
  
  \item
    Semesterwoche Keynote mit golive
  \end{enumerate}
\end{itemize}

\hypertarget{aufgaben-bis-nachstes-meeting}{%
\subsection*{Aufgaben bis nächstes
Meeting}\label{aufgaben-bis-nachstes-meeting}}

\hypertarget{prof.-dr.-olaf-zimmermann}{%
\subsubsection*{Prof. Dr. Olaf
Zimmermann}\label{prof.-dr.-olaf-zimmermann}}

\hypertarget{elias-ux5cux26-oli}{%
\subsubsection*{Elias \& Oli}\label{elias-ux5cux26-oli}}

\begin{itemize}

\item
  Design thinking youtube terminologie
\item
  (APF Wiki) \href{https://wiki.ifs.hsr.ch/APF/wiki.cgi?UebSpezialMAP}{MAP} lesen
\item
  Zusage für BA
\end{itemize}