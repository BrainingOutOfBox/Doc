\hypertarget{meeting-notes-27.09.2018}{%
\section*{Meeting notes 27.09.2018}\label{meeting-notes-27.09.2018}}

\hypertarget{anwesende}{%
\subsection*{Anwesende}\label{anwesende}}

\begin{itemize}

\item
  Prof. Dr. Olaf Zimmermann
\item
  Elias Brunner
\item
  Oliver Dias
\end{itemize}

\hypertarget{agenda}{%
\subsection*{Agenda}\label{agenda}}

\hypertarget{time-tracking}{%
\subsubsection*{Time tracking}\label{time-tracking}}

\begin{itemize}

\item
  Reicht eine Auswertung von beiden zusammen? Nein
\item
  Müssen wöchentliche Reports/Aussagen über aufgewendete Stunden gemacht
  werden?
\item
  Zeitauswertung möglich geordnet nach Benutzer, Vorgang, etc.
  -\textgreater{} 5 bis 10 Labels, nach deren Zeitabrechnung gemacht
  wird (Richtwert PM: ca. 10\%) sowie individuelle Zeitabrechnung
\item
  Im Anhang der Doku dann Zeitauswertung, wird sporadisch nachgefragt
\end{itemize}

\hypertarget{unterschriften}{%
\subsubsection*{Unterschriften}\label{unterschriften}}

\begin{itemize}

\item
  Müssen all die unterschriebenen Dokumente, Projektauftrag und
  Protokolle in der Hauptabgabe sein? (inkl Inhaltsverzeichnis)
\end{itemize}

-\textgreater{} Einverständniserklärung kommt in Bericht

\hypertarget{repo-auf-private-stellen}{%
\subsubsection*{Repo auf private
stellen}\label{repo-auf-private-stellen}}

\begin{itemize}

\item
  Wie wichtig ist das umstellen auf private?
\item
  \href{https://sonarcloud.io/about}{SonarQube} kann nur auf Open Source
  Projekte gratis angewendet werden.
\item
  GitHub hat unseren Antrag abgelehnt.
\end{itemize}
\textbf{Entscheid}: Alles was von Betreuer kommt muss private sein.
-\textgreater{} Meeting notes auf switchdrive. Code \& Doku kann public
bleiben.

\hypertarget{review-projektplan}{%
\subsubsection*{Review Projektplan}\label{review-projektplan}}

\begin{itemize}
\item
  Was ist das Feedback?
\item
  Review von Protokoll und Projektplan
\item
  Wenn Copy Paste von anderen Projekte/Quellen und/oder einzelne Wörter
  ausgetauscht -\textgreater{} Angabe in BibTeX
\item
  Projektrisiken projektspezifischer formulieren
\item
  Projektplan auch zu allgemein
\item
  Auf Board geschaut. Rückmeldung: für Kategorie "Design" Labels
  erstellen und Tickets dazu hinzufügen
\end{itemize}

\hypertarget{azure-ux2f-app-center}{%
\subsubsection*{Azure / App Center}\label{azure-ux2f-app-center}}

\begin{itemize}

\item
  Würde uns viel Zeit sparen, da build schon funktioniert ohne grossen
  Aufwand.
\item
  Aus Erfahrung ist das Aufsetzten von TravisCI/Jenkins sehr
  zeitaufwändig (1-2 Tage).
\item
  4h Build Zeit pro Monat
\item
  Das automatische Ausführen von Unit Tests sind wir am Aufsetzen
  -\textgreater{} Lizenz für App Center von Azure klärt Herr Zimmermann
  ab.
\end{itemize}

\hypertarget{backend}{%
\subsubsection*{Backend}\label{backend}}

\begin{itemize}

\item
  Aus eigener Erfahrung kenne ich das PlayFramework relativ gut.
\item
  Gibt es Frameworks, welche sonst noch erfohlen werden können?
\end{itemize}

Zwei Tipps:

\begin{enumerate}
\def\labelenumi{\arabic{enumi}.}

\item
  Firebase, nur DB
\item
  Deutsches Startup: Backend As A Service von \href{https://vsis-www.informatik.uni-hamburg.de/getDoc.php/publications/522/Towards\%20a\%20Scalable\%20and\%20Unified\%20REST\%20API\%20for\%20Cloud\%20Data\%20Stores.pdf}{Uni Hamburg}
\item
  Auch Play-Framework empfehlenswert
\end{enumerate}

Empfehlung: auf gleicher Technologie wie Client (C\#), aber REST
Schnittstellen sind interoperabel. Nicht mehr als 2-3 h in Research
aufwenden.

\hypertarget{aufgaben-bis-nachstes-meeting}{%
\subsection*{Aufgaben bis nächstes
Meeting}\label{aufgaben-bis-nachstes-meeting}}

\hypertarget{prof.-dr.-olaf-zimmermann}{%
\subsubsection*{Prof. Dr. Olaf
Zimmermann}\label{prof.-dr.-olaf-zimmermann}}

\begin{itemize}

\item
  Abklären wegen Azure Lizenz für MS App Center (uneingeschränkte
  Buildminuten)
\end{itemize}

\hypertarget{elias-ux5cux26-oli}{%
\subsubsection*{Elias \& Oli}\label{elias-ux5cux26-oli}}

\begin{itemize}

\item
  Review von Projektplan einarbeiten.
\item
  Einverständniserklärung kommt in Bericht
\item
  Meeting Traktanden auch im Voraus schicken
\item
  Andere Berichte und Bewertungskriterien durchgehen
\item
  Switchdrive für Meetingnotes einrichten und Ordner sharen. Dennoch
  E-mail mit Notes im Anhang!
\item
  Meetingleiter: Anfang des Meetings Grobüberblick des Projektverlaufs
\end{itemize}