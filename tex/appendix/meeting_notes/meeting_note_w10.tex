\hypertarget{meeting-notes-22.11.2018}{%
\section*{Meeting notes 22.11.2018}\label{meeting-notes-22.11.2018}}

\hypertarget{anwesende}{%
\subsection*{Anwesende}\label{anwesende}}

\begin{itemize}

\item
  Prof. Dr. Olaf Zimmermann
\item
  Elias Brunner
\item
  Oliver Dias
\end{itemize}

\hypertarget{agenda}{%
\subsection*{Agenda}\label{agenda}}

\hypertarget{organisatorisches}{%
\subsubsection*{Organisatorisches}\label{organisatorisches}}

\begin{itemize}

\item
  Vorschlag für User-Test mit Testprotokoll am 06.12
\item
  Einen kompletten Durchgang machen und gleichzeitig User-Test
  durchführen 
\item \textbf{FEEDBACK}
\item
  Für solche Tests gibt es Templates aus anderen Modulen
\item
  Klingt vernünftig
\item
  Testprotokoll daher ausführlicher dafür aber Unit Tests kleiner
\end{itemize}

\hypertarget{catch-upux2fstatus}{%
\subsubsection*{Catch-up/Status}\label{catch-upux2fstatus}}

\begin{itemize}

\item
  Zeigen, was alles im Frontend (Backend) gemacht wurde
\item
  Grosser Fortschritt im Frontend seit dem letzten Mal -\textgreater{}
  Gut in der Zeit
\item
  Blatt-Wechsel-Logik im Frontend noch umsetzen
\end{itemize}

\hypertarget{aufgaben-bis-nachstes-meeting}{%
\subsection*{Aufgaben bis nächstes
Meeting}\label{aufgaben-bis-nachstes-meeting}}

\hypertarget{prof.-dr.-olaf-zimmermann}{%
\subsubsection*{Prof. Dr. Olaf
Zimmermann}\label{prof.-dr.-olaf-zimmermann}}

\begin{itemize}

\item
  Genaue Abklärung wegen Bachelor-Arbeiten
\end{itemize}

\hypertarget{elias-ux5cux26-oli}{%
\subsubsection*{Elias \& Oli}\label{elias-ux5cux26-oli}}

\begin{itemize}

\item
  DI W13 das gesammte Dokument zum Review abgeben
\item
  Wir organisieren ein 3. Smartphone aber das Institut für Software
  hätte bestimmt auch noch welche
\end{itemize}