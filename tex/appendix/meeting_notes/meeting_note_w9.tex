\hypertarget{meeting-notes-15.11.2018}{%
\section*{Meeting notes 15.11.2018}\label{meeting-notes-15.11.2018}}

\hypertarget{anwesende}{%
\subsection*{Anwesende}\label{anwesende}}

\begin{itemize}

\item
  Prof. Dr. Olaf Zimmermann
\item
  Elias Brunner
\item
  Oliver Dias
\end{itemize}

\hypertarget{agenda}{%
\subsection*{Agenda}\label{agenda}}

\hypertarget{organisatorisches}{%
\subsubsection*{Organisatorisches}\label{organisatorisches}}

\begin{itemize}

\item
  Forsetzung BA -\textgreater{} Beidseitig zugesagt!
\item
  Rückmeldung Formatierung Meeting Notes
\end{itemize}

\hypertarget{review-doku}{%
\subsubsection*{Review Doku}\label{review-doku}}

\begin{itemize}

\item
  \grqq Die Gründe können\grqq{} umformulieren
\item
  \grqq Ausschnitt aus der Klasse\grqq{} hinzufügen für besseren Kontext
\item
  Umbruch optimieren, Hyphen nächste Zeile
\item
  Allgemeine Coding-Conventions einhalten!
\item
  zu viel passiv, besser: "Der ParticipantController sendet zurück",
  "Das Listing zeigt"
\item
  etwas kompakter, Leerzeilen im Code vermeiden
\item
  überlappende Zeile umbrechen (-\textgreater{} im LaTeX)
\item
  Alle Abbildungen und Tabellen müssen im Text erklärt und erwähnt
  werden (editorielle Auflage)
\item
  recht detailliert -\textgreater{} gut so, nicht mehr genauer
\item
  zu Beginn dieser Abschnitte Rückwärtsreferenz auf betroffene
  Komponente
\item
  Begründung zu Beginn, wieso genau diese Komponenten genauer erklärt
  sind
\end{itemize}

\hypertarget{catch-upux2fstatus}{%
\subsubsection*{Catch-up/Status}\label{catch-upux2fstatus}}

\begin{itemize}

\item
  Zeigen, was alles im Front- bzw. Backend gemacht wurde
\end{itemize}

\hypertarget{testing-konzept-erlautern}{%
\subsubsection*{Testing Konzept
erläutern}\label{testing-konzept-erlautern}}

\begin{itemize}

\item
  Testziele festelgen (Benutzbarkeit, Robustheit, Umsetzung fachliche
  Anforderung)
\item
  Testkonzept (1 Seite)
\item
  kleiner Bericht: was kam raus? -\textgreater{} Pragmatisch
\end{itemize}

\hypertarget{user-tests}{%
\subsubsection*{User Tests}\label{user-tests}}

\begin{itemize}

\item
  Zielsetzung: Bis Ende Projekt einen Durchlauf präsentieren können (mit
  iPhone Simulator auf Elias' Laptop + Android von Oli)
\end{itemize}

\hypertarget{aufgaben-bis-nachstes-meeting}{%
\subsection*{Aufgaben bis nächstes
Meeting}\label{aufgaben-bis-nachstes-meeting}}

\hypertarget{prof.-dr.-olaf-zimmermann}{%
\subsubsection*{Prof. Dr. Olaf
Zimmermann}\label{prof.-dr.-olaf-zimmermann}}

\hypertarget{elias-ux5cux26-oli}{%
\subsubsection*{Elias \& Oli}\label{elias-ux5cux26-oli}}