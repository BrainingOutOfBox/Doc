\hypertarget{meeting-notes-11.10.2018}{%
\section*{Meeting notes 11.10.2018}\label{meeting-notes-11.10.2018}}

\hypertarget{anwesende}{%
\subsection*{Anwesende}\label{anwesende}}

\begin{itemize}

\item
  Prof. Dr. Olaf Zimmermann
\item
  Elias Brunner
\item
  Oliver Dias
\end{itemize}

\hypertarget{agenda}{%
\subsection*{Agenda}\label{agenda}}

\hypertarget{aufgabenstellung}{%
\subsubsection*{Aufgabenstellung}\label{aufgabenstellung}}

\begin{itemize}

\item
  Einzelne Punkte besprechen; Was heisst das konkret
\item
  Links anstatt Video
\item
  Notiz an uns: im Projektplan einen Dokumentenplan erstellen. Welche
  Dokumente müssen abgegeben werden.
\item
  Aufgabenstellung dev. festlegen
\end{itemize}

\textbf{Entscheide:}

\begin{itemize}

\item
  Punkte können protokoiert werden (lassen sich so nicht ausarbeiten wie
  geschrieben) oder ganz gelöscht werden.
\item
  Hiermit halten wir fest, dass die entgültige Abgabe (App) nicht
  komplett ortsunabhänig sein wird und auch die Teilnehmer ihre
  Lösungsvorschläge nicht zeitunabhängig erarbeiten können.
\end{itemize}

\hypertarget{review-domain-modell}{%
\subsubsection*{Review Domain Modell}\label{review-domain-modell}}

\begin{itemize}

\item
  Bei der Beschreibung \grqq erweiterbar\grqq{} ersetzen -\textgreater{} wie z.B.
  (sehen wir) ...
\end{itemize}

\hypertarget{review-mockup}{%
\subsubsection*{Review Mockup}\label{review-mockup}}

\begin{itemize}

\item
  Soll die Länge der Texte beschränkt werden? (besseres UX)
\item
  Soll es Highlighting der Texte geben?
\item
  Landscape oder Portrait für die Screens?
\item
  Gute User Meldungen (Rückmeldungen) Feedback geben -\textgreater{}
  Beim Vorgänger Blatt leer -\textgreater{} Dein Vorgänger hat nichts
  geschrieben
\item
  New Brainstroming statt New Problem
\item
  Number of Participants -\textgreater{} first come first serve;
  Irgendwo schreiben
\item
  Sprache für UI soll Englisch sein aber die Möglichkeit bieten auf
  ander Sprachen zu wechseln
\item
  Soll zu Beginn die Möglihckeit bestehen, dass der Moderator zuerst
  eine Begrüssung macht. Er erklärt die Methode (Spielregeln erklären)
  zuerst.
\end{itemize}

\hypertarget{architekturfragen}{%
\subsubsection*{Architekturfragen}\label{architekturfragen}}

\begin{itemize}

\item
  Logische Schichtenarchitektur für das Gesamtsystem oder
  Backend/Frontend separat?
\end{itemize}

\textbf{Entscheide}:
\begin{itemize}

\item
  Zuerst eine Sicht der Schichten zu einem späteren Zeitpunkt
  verfeinern.
\item
  Client-Server Cut definieren
\item
  Timer -\textgreater{} Tier 1 Timer (TimerClient), Tier2 Timer
  (TimerSync) -\textgreater{} Distributed Application Kernel
\item
  BLL: Use Cases und DomainModell sind Inputs -\textgreater{} ein
  Diagramm,
\item
  Presentation Layer: BrainstromingFindingView, BrainsheetView,
  BrainwaveView, etc.
\end{itemize}

\hypertarget{prototype}{%
\subsubsection*{Prototype}\label{prototype}}

\begin{itemize}

\item
  Als erstes das Timemanagement (BrainwaveFindings) umsetzen
\end{itemize}

\hypertarget{kommunikation-zwischen-server-und-app}{%
\subsubsection*{Kommunikation zwischen Server und
App}\label{kommunikation-zwischen-server-und-app}}

\begin{itemize}

\item
  NAT besser umschreiben auf IP-Ebene
\item
  NAT ist zu knapp bessere Umschreibung
\item
  Firewalls, Authenication-Proxies sind eher das Problem
\end{itemize}

\hypertarget{typo}{%
\subsubsection*{Typo}\label{typo}}

\begin{itemize}

\item
  Polling nicht Pulling
\item
  4.7.4 Titel zu Peer zu Peer ändern
\end{itemize}

\hypertarget{aufgaben-bis-nachstes-meeting}{%
\subsection*{Aufgaben bis nächstes
Meeting}\label{aufgaben-bis-nachstes-meeting}}

\hypertarget{prof.-dr.-olaf-zimmermann}{%
\subsubsection*{Prof. Dr. Olaf
Zimmermann}\label{prof.-dr.-olaf-zimmermann}}

\hypertarget{elias-ux5cux26-oli}{%
\subsubsection*{Elias \& Oli}\label{elias-ux5cux26-oli}}