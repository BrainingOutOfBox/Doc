\hypertarget{meeting-notes-29112018}{%
\section*{Meeting notes 29.11.2018}\label{meeting-notes-29112018}}

\hypertarget{anwesende}{%
\subsection*{Anwesende}\label{anwesende}}

\begin{itemize}

\item
  Prof. Dr. Olaf Zimmermann
\item
  Elias Brunner
\item
  Oliver Dias
\end{itemize}

\hypertarget{agenda}{%
\subsection*{Agenda}\label{agenda}}

\hypertarget{organisatorisches}{%
\subsubsection*{Organisatorisches}\label{organisatorisches}}

\begin{itemize}
\item
  Muss das Titelblatt genau so aussehen wie die Vorlage?
\item
  Die Aufgabenstellung haben wir ihnen zur Unterschrift gegeben. Diese
  müsste unterschrieben in den Bericht.
\item
  Auch die Urheber- und Nutzungsrechte müssen unterschrieben in den
  Bericht
\item
  Smartphone organisiert \& lauffähig
\item
  Installationsanleitung
\item
  Abbildungsverzeichnis nötig?
\item
  HSR Logo muss auf Titelblatt, inhaltsgemäss nach Vorlage
\item
  Abbildungsverzeichnis wird gern gesehen. Hinter Literaturverzeichnis
  mit Abbildungsnummer, Bezeichnung und Seitenzahl
\item
  Aufgabenstellung und Urhebernutzungsrecht sind bei Herrn Zimmermann
\item
  Installationsanleitung: Anleitung fürs Aufsetzen, sodass lokal Server
  und App läuft (inklusive Verweis auf Xamarin
  Installationsdokumentation)
\end{itemize}

\hypertarget{feedback-bericht}{%
\subsubsection*{Feedback Bericht}\label{feedback-bericht}}

\begin{itemize}
\item
  Kapitel 4.1, 4.2, 4.8.4 lesen
\item
  Erster Satz 4.1: \grqq den Grossteil\grqq störend, besser rausnehmen
\item
  Übernächster Satz: \grqq Stellen Einführung dar\grqq, Wiederholung nicht
  zweimal in drei Sätzen, aber durch Wegnehmen obigen Punktes löst sich
  dieses Problem
\item
  \grqq All dies ist niedergeschrieben\grqq -\textgreater{} etwas ungewöhnlich
\item
  Tipp: Selbst laut vorlesen, oder andere Person lesen lassen
\item
  4.2 Wichtig, dass Mono ins Leben gerufen wurde? Vlt erstes Vorkommen
  rausnehmen, danach an relevanter Stelle reinnehmen
\item
  Bei einleitenden Kapitel sich fragen: Für wen schreibe ich? Was wissen
  Leser? Was wissen die Leser bereits im aktuellen Abschnitt?
\item
  4.8.4 MongoDB: \grqq Document\grqq Englisch, \grqq zusammengehörig zusammen\grqq (besser
  zB. gemeinsam gespeichert). Referenzen gut gemacht.
\item
  Wenn Hauptgrund erwähnt, dann auch Nebengrund erwartet.
\item
  Begründung Wahl MongoDB: Mächtigkeit von relationalen DBs nicht nötig,
  keine komplizierte Auswertung. Dokumentorientiertes Paradigma reicht
  somit aus.
\item
  Als Erweiterung noch: Wieso Mongo und nicht Couch? (Vorkommen im
  Studium etc)
\item
  Auch Analog für andere Entscheidungen
\item
  Einleitung Technischer Bericht: Referenzen auf die verschiedenen
  Kapitel können gemacht werden
\end{itemize}

\hypertarget{catch-upstatus}{%
\subsubsection*{Catch-up/Status}\label{catch-upstatus}}

\begin{itemize}

\item
  Falls wirklich benötigt, besteht Möglichkeit auf Verschieben des
  Testtermins
\item
  Vorläufig Termin beibehalten, ansonsten in nächsten Traktanden
  Bescheid geben
\end{itemize}

\hypertarget{aufgaben-bis-nuxe4chstes-meeting}{%
\subsection*{Aufgaben bis nächstes
Meeting}\label{aufgaben-bis-nuxe4chstes-meeting}}

\hypertarget{prof-dr-olaf-zimmermann}{%
\subsubsection*{Prof. Dr. Olaf
Zimmermann}\label{prof-dr-olaf-zimmermann}}

\begin{itemize}

\item
  Aufgabenstellung und Urhebernutzungsrecht mitbringen für Scan
\end{itemize}

\hypertarget{elias--oli}{%
\subsubsection*{Elias \& Oli}\label{elias--oli}}

\begin{itemize}

\item
  In den nächsten Traktanden mitteilen, ob Test stattfinden kann oder
  nicht
\end{itemize}