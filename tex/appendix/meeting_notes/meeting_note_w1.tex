\hypertarget{meeting-notes-20.09.2018}{%
\section*{Meeting notes 20.09.2018}\label{meeting-notes-20.09.2018}}

\hypertarget{anwesende}{%
\subsection*{Anwesende}\label{anwesende}}

\begin{itemize}

\item
  Prof. Dr. Olaf Zimmermann
\item
  Elias Brunner
\item
  Oliver Dias
\end{itemize}

\hypertarget{agenda}{%
\subsection*{Agenda}\label{agenda}}

\hypertarget{generelle-punkte}{%
\subsubsection*{Generelle Punkte}\label{generelle-punkte}}

\begin{itemize}

\item
  Projektleiter Position abwechseln
\item
  Meeting wird ab nächster Woche von uns geleitet
\item
  GitHub: @socadk
\item
  Meeting Notes als md Mail im Anhang
\item
  24h vor Meeting Dokument/geleistete Arbeiten abgeben für Review
\item
  Standardmässig findet Meeting Donnerstags um 9 Uhr statt, in Absprache
  auch späterere Termine am Vormittag möglich
\end{itemize}

\hypertarget{randbedingungen}{%
\subsubsection*{Randbedingungen}\label{randbedingungen}}

\begin{itemize}

\item
  Abgabe Projekt: Heraufladen auf archiv-i.hsr.ch 21.12.2018 12:00 Uhr
\item
  Allgemeine Termine befinden sich auf SA-Website/SkripteServer/Intranet
  Studien/Bachelorarbeit (siehe useful links)
\item
  Benotungskriterien nächstes mal genauer, Betreuer hält sich an Vorgabe
\end{itemize}

\hypertarget{review-aufgabenstellung}{%
\subsubsection*{Review Aufgabenstellung}\label{review-aufgabenstellung}}

\begin{itemize}

\item
  Formulierung hat noch Luft noch oben
\item
  Berichte werden in aktiv bevorzugt
\item
  Technisches Schreiben Buch: unter
  \href{https://www.ifs.hsr.ch/index.php?id=13194\&L=4campusmap\%2F}{ifs.hsr.ch},
  \href{https://link.springer.com/book/10.1007\%2F978-3-642-13827-0}{Texte
  für die Technik}
\end{itemize}

\hypertarget{documentation-review}{%
\subsubsection*{Documentation Review}\label{documentation-review}}

\begin{itemize}

\item
  3 mal Story erzählen -\textgreater{} Abstract, Management Summary,
  Bericht selbst
\item
  Überschrift "Technischer Bericht" ändern evtl auf: Anforderung,
  Konzept und Umsetzung
\item
  Usertests/Logs dann auch in Anhang
\item
  Einleitung/Übersicht verschmelzen
\item
  Keine Abschnitte mit nur einem Unterabschnitt
\item
  Nicht über 8/9 Unterkapitel, ansonsten in Anhang
\item
  Domainanalyse -\textgreater{} Anforderungsanalyse in einem
\item
  \textbf{Inhaltscheckliste} (normativ) auf Anleitung Dokumentation
\item
  NFA, Qualitätsattribute noch reinnehmen
\item
  Anstelle von Ergebnisse: Architektur und Implementierung
\item
  Mit Überschrift kann der Leserschaft bekanntgemacht werden, für wen
  dieser Abschnitt geeignet ist.
\end{itemize}

\hypertarget{azure-license}{%
\subsubsection*{Azure License}\label{azure-license}}

\begin{itemize}

\item
  Besser nicht proprietär
\item
  \textbf{Entscheid:} darauf verzichten solange nicht spezifisch
  begründet wird, wieso nur genau Azure gesuchte Funktionalität bietet
\end{itemize}

\hypertarget{management-methodik}{%
\subsubsection*{Management Methodik}\label{management-methodik}}

\begin{itemize}

\item
  \emph{Tipp}: Einzelne Elemente von RUP, Scrum, Wasserfall heraussuchen
  und verwenden
\item
  Generell aber Scrum/Agile
\end{itemize}

\hypertarget{infrastructure-ux5cux26-tools}{%
\subsubsection*{Infrastructure \&
Tools}\label{infrastructure-ux5cux26-tools}}

\begin{itemize}

\item
  Planung mit Soll/Ist Vergleich
\item
  Time Tracking
\item
  Als Anhang im Bericht
\item
  Jira iO
\item
  App Center iO
\end{itemize}

\hypertarget{weitere-wichtige-punkte}{%
\subsubsection*{Weitere wichtige Punkte}\label{weitere-wichtige-punkte}}

\begin{itemize}

\item
  Projektverlauf: NFA sauber spezifizieren (spezifisch \& messbar,
  überprüfbar) 7-10 reichen erstmal
\item
  in eprints bereits betreute Arbeiten, als Vorbild anschauen
\item
  Architekturentscheide begründen und beschreiben, welhalb dieses
  Design, welche Patterns etc.
\item
  Editorielle Qualität wichtig
\end{itemize}

\hypertarget{tasks-until-next-meeting}{%
\subsection*{Tasks until next meeting}\label{tasks-until-next-meeting}}

\hypertarget{prof.-dr.-olaf-zimmermann}{%
\subsubsection*{Prof. Dr. Olaf
Zimmermann}\label{prof.-dr.-olaf-zimmermann}}

\begin{itemize}

\item
  Bewertungskriterien anschauen
\end{itemize}

\hypertarget{elias-ux5cux26-oli}{%
\subsubsection*{Elias \& Oli}\label{elias-ux5cux26-oli}}

\begin{itemize}

\item
  Einverständniserklärung unterschreiben
\item
  Unseren Strukturierungsvorschlag mit Inhaltscheckliste vom
  Skripteserver anpassen
\item
  Welche Methodenelemente sollen verwendet werden von Scrum, RUP
\item
  Repos auf private umstellen
\end{itemize}