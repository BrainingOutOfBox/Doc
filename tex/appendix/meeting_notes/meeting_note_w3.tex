\hypertarget{meeting-notes-04.10.2018}{%
\section*{Meeting notes 04.10.2018}\label{meeting-notes-04.10.2018}}

\hypertarget{anwesende}{%
\subsection*{Anwesende}\label{anwesende}}

\begin{itemize}

\item
  Prof. Dr. Olaf Zimmermann
\item
  Elias Brunner
\item
  Oliver Dias
\end{itemize}

\hypertarget{agenda}{%
\subsection*{Agenda}\label{agenda}}

\hypertarget{einleitung}{%
\subsubsection*{Einleitung}\label{einleitung}}

\begin{itemize}

\item
  Zitat Methode 635 etwas zu lang, besser Zusammenfassen mit gleicher
  Einleitung
\end{itemize}

\hypertarget{use-case-diagramm}{%
\subsubsection*{Use Case Diagramm}\label{use-case-diagramm}}

\begin{itemize}
\item
  Diagramm besprechen
\item
  View Result -\textgreater{} View Innovation Ideas
\item
  Rundenschritte können nicht herausgelesen werden
\item
  Multimedia/Links wo kommen diese im UC?
\item
  Weiterer Abuse Case: Gruppengeheimnisse erfahren
\item
  Group evtl umschreiben auf zB Brainstorminggroup, etwas
  projekt-/domänenspezifischer
\item
  Gesamteindruck gut
\end{itemize}

\hypertarget{ablaufdiagramm-ux2f-sequenzdiagramm}{%
\subsubsection*{Ablaufdiagramm /
Sequenzdiagramm}\label{ablaufdiagramm-ux2f-sequenzdiagramm}}

\begin{itemize}
\item
  Unsere Überlegungen aufzeigen
\item
  Polling ist unschön. Gib es eine bessere Lösung?
\item
  Runden mit Timer lösen oder gibt es auch hier eine elegantere Lösung?
\item
  Polling: Mit Landing Zones arbeiten (aggressive, moderate und einfache
  Anz. Requests) für Architekturentscheid aufschreiben NFR
\item
  Aktuellem Stadium iO, evtl später etwas detailierter
\end{itemize}

\hypertarget{review-nicht-funktionale-anforderungen}{%
\subsubsection*{Review Nicht-Funktionale
Anforderungen}\label{review-nicht-funktionale-anforderungen}}

\begin{itemize}
\item
  Feedback durch Herrn Zimmermann
\item
  ISO 25010 (NFR) Quellenverweis fehlt
\item
  "Grössenordnung von" nicht maximal/unter, begründen
\item
  Als Einleitung, "folgendes gilt für alle UCs"
\item
  Für Backend NFRs?
\end{itemize}

\hypertarget{backend}{%
\subsubsection*{Backend}\label{backend}}

\begin{itemize}
\item
  Haben uns ca. 30min in \href{baqend.com}{baqend.com} eingearbeitet.
\item
  Prinzipiell vielversprechend doch bei eigener Logik ist man wohl rasch
  am Anschlag.
\item
  Restful webservices cookbook Buchempfehlung
\end{itemize}

\hypertarget{aufgaben-bis-nachstes-meeting}{%
\subsection*{Aufgaben bis nächstes
Meeting}\label{aufgaben-bis-nachstes-meeting}}

\hypertarget{prof.-dr.-olaf-zimmermann}{%
\subsubsection*{Prof. Dr. Olaf
Zimmermann}\label{prof.-dr.-olaf-zimmermann}}

\hypertarget{elias-ux5cux26-oli}{%
\subsubsection*{Elias \& Oli}\label{elias-ux5cux26-oli}}

\begin{itemize}

\item
  Aufgabenstellung nochmals durchgehen und nächstes Meeting mitnehmen
  (V1.0 unbennen)
\item
  Erwähnen, welche Seiten genau besprochen werden sollen
\item
  Notizen mit Rückmeldung wieder mitnehmen
\item
  Christian Spielmann Mail, mit ZIO als Mail, mit genauem Link mit was
  wir haben wollen mit Begründung (gratis reicht nicht aus um effizient
  zu arbeiten, für Projekterfolg wichtig mit Verweis auf SE? Kosten auch
  aufschreiben (Fixkosten) Mit Betreuer abgesprochen Studiengangleitung.
  Fragen an uns oder ZIO anrufen).
\end{itemize}