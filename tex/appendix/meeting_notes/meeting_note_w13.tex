\hypertarget{meeting-notes-13122018}{%
\section*{Meeting notes 13.12.2018}\label{meeting-notes-13122018}}

\hypertarget{anwesende}{%
\subsection*{Anwesende}\label{anwesende}}

\begin{itemize}

\item
  Prof. Dr. Olaf Zimmermann
\item
  Elias Brunner
\item
  Oliver Dias
\end{itemize}

\hypertarget{agenda}{%
\subsection*{Agenda}\label{agenda}}

\hypertarget{organisatorisches}{%
\subsubsection*{Organisatorisches}\label{organisatorisches}}

\begin{itemize}
\item
  Zeitauswertung für die letzte Woche
\item
  Mit Durchschnittswerten rechnen
\item
  NFA Wiederverwendbarkeit - keine SonarQube Auswertung, die duplicated
  Code zeigt (nur Linter in IDE installiert). Stellungsnahme oder so
  belassen?
\item
  Auch Umformulierung möglich -\textgreater{} rausnehmen/anderes
  Kriterium
\end{itemize}

\hypertarget{feedback-bericht}{%
\subsubsection*{Feedback Bericht}\label{feedback-bericht}}

\begin{itemize}

\item
  Logo kleiner
\item
  IFS \& HSR Logo
\item
  Grösster Mangel ist sprachliches
\item
  Genauerer Vergleich Paper/App
\item
  Wenn möglich: mit gleichem Team wie bei Vorstudie mit App Testrunde
  durchführen und bewerten
\item
  Stärken/Schwächen ausführlicher
\item
  Ausblick konkreter -\textgreater{} Richtung Bachelorarbeit, Software
  Entwurf/Design Thinking
\item
  Aufgabenstellung als Hr. Zimmermanns Arbeit kennzeichnen
\item
  keine Nummer bei Aufgabenstellung \& Abstract
\item
  Management Summary positiv enden
\item
  Allgemein: Leser besser abholen
\end{itemize}

\textbf{Herausforderungen}

\begin{itemize}

\item
  Während des Entwickelns
\item
  nicht mehr -\textgreater{} Version angeben
\end{itemize}

\textbf{Kritisches Design der Brainstorming Logik}

\begin{itemize}

\item
  Andere Überschrift, oder andere Überschrift im HTTPS
\item
  Kritisches \textless{}-\textgreater{} Anspruchsvolles/komplexes
\item
  Bild erklären -\textgreater{} Talk reader through figure
\end{itemize}

\textbf{Implementation PlayFramework}

\begin{itemize}

\item
  Verwendete Bibliotheken im \emph{Backend}
\end{itemize}

\textbf{Implementation Xamarin}

\begin{itemize}

\item
  MVVM in ein bis zwei Sätzen erklären
\item
  Three letter acronym
\item
  Verweis kann auch reichen
\item
  Dependency Injection auch genauer, siehe Ref
\item
  Swipe Gesture: "andere Zwecke" erklären!
\item
  ZXing: genauer erklären wieso diese Library evtl in Architectural
  Decision
\item
  Abschnitt muss Text haben! (Tabellen mit Libraries)
\end{itemize}

\textbf{Vergleich Soll/Ist}

\begin{itemize}

\item
  Die meisten Ziele zu ungenau, Ziele wurden wie folgt erreicht
\item
  zusammenfassend 8/10 Ziele erreicht
\item
  Persistierung: Anbieten dass per HTTP-Get den Brainstormingfinding
  Dump geholt werden kann. Als Ausblick?
\item
  Nicht intuitiv -\textgreater{} Wer sagt das? Tests? Stärker wie zu den
  Schlüssen gekommen. Fakten und Interpretationen trennen
\end{itemize}

\textbf{Schlussfolgerungen}

\begin{itemize}

\item
  Umgangssprachliches
\item
  Kapitel .. zeigt auf/behandelt
\item
  Abbildungen/Tabellen agieren lassen anstelle von passiv
\item
  Wie Tests ergaben/Self assessment?
\item
  Ähnliches Vorgehen zu ungenau
\end{itemize}

\textbf{Testprotokoll}

\begin{itemize}

\item
  Auf diese Tests im Hauptteil verweisen (von Schlussfolgerung/Vergleich
  Soll/Ist)
\item
  Testsetup aufzeigen
\item
  Column Beobachtetes Verhalten anstelle von Bestanden J/N
\item
  Noch textuell zusammenfassen
\item
  Genauer erklären, was ist mit Führung gemeint?
\end{itemize}

\hypertarget{demo}{%
\subsubsection*{Demo}\label{demo}}

\hypertarget{aufgaben-bis-nuxe4chstes-meeting}{%
\subsection*{Aufgaben bis nächstes
Meeting}\label{aufgaben-bis-nuxe4chstes-meeting}}

\hypertarget{prof-dr-olaf-zimmermann}{%
\subsubsection*{Prof. Dr. Olaf
Zimmermann}\label{prof-dr-olaf-zimmermann}}

\hypertarget{elias--oli}{%
\subsubsection*{Elias \& Oli}\label{elias--oli}}

\begin{itemize}

\item
  Abgabe mit USB Stick 2 Versionen Doku (inkl. SourceCode FE/BE)
\end{itemize}