\hypertarget{meeting-notes-18.10.2018}{%
\section*{Meeting notes 18.10.2018}\label{meeting-notes-18.10.2018}}

\hypertarget{anwesende}{%
\subsection*{Anwesende}\label{anwesende}}

\begin{itemize}

\item
  Prof. Dr. Olaf Zimmermann
\item
  Elias Brunner
\item
  Oliver Dias
\end{itemize}

\hypertarget{agenda}{%
\subsection*{Agenda}\label{agenda}}

\hypertarget{rest-api-ux2f-app-prototype}{%
\subsubsection*{REST-API / App
Prototype}\label{rest-api-ux2f-app-prototype}}

\begin{itemize}

\item
  Verschlüsselung wegen iOS
\item
  Beide am Laufen haben (gleichzeitiges Herunterzählen)
\item
  \grqq Blatt-Wechsel-Logik\grqq{} implementieren
  
\item
  \textbf{Feedback}: Welche Risiken sind eingetreten -\textgreater{} ssl
\end{itemize}

\hypertarget{feedback-diagramme}{%
\subsubsection*{Feedback Diagramme}\label{feedback-diagramme}}

\begin{itemize}

\item
  Deployment Diagramm besprechen
  \item \textbf{Feedback}
\item
  Mit Legende und Text
\item
  statt server -\textgreater{} backendNode, auf dem Node sind
  deploymendUnits (DU)
\item
  evtl. client links, presentation eine box (presentationLayerDU),
  business logic eine box (businessLayerDU)
\item
  Architekturdiagramm besprechen
\item \textbf{Feedback}
\item
  Legende mit Text
\item
  Bessere Begriffe wählen. Statt Backend/Frontend
\item
  Statt PersitenceComp -\textgreater{} DataAccessComponent
\item
  Im Backend ist die DB auch enthalten
\item
  DataStore hinzufügen von PersistenceComponent, Participants, Team,
  etc.
\end{itemize}

\hypertarget{input-zum-testing}{%
\subsubsection*{Input zum Testing}\label{input-zum-testing}}

\begin{itemize}

\item
  Wie sieht das allgemeine Testvorgehen aus?
\item \textbf{Feedback}
\item
  Zielgetriebenes Testing
\item
  Postman oder curl
\item
  Für jede Component ein Testfall (Was ist das Ziel?)
\end{itemize}

\hypertarget{azure}{%
\subsubsection*{Azure}\label{azure}}

\begin{itemize}

\item
  Weiss Herr Spielmann wirklich, dass wir das OK erhalten haben?
\item
  Schnell vorbei gehen
\end{itemize}

\hypertarget{meeting-am-01.11}{%
\subsubsection*{Meeting am 01.11}\label{meeting-am-01.11}}

\begin{itemize}

\item
  Wie sieht es am 01.11 aus?
\item
  Ist Feiertag, wäre kein Meeting
\item
  Am Dienstag wäre Herr Zimmermann aber an der HSR
\item
  Nächste Woche thematisieren, ob anderer Termin oder ausfallen lassen
\end{itemize}

\hypertarget{aufgaben-bis-nachstes-meeting}{%
\subsection*{Aufgaben bis nächstes
Meeting}\label{aufgaben-bis-nachstes-meeting}}

\hypertarget{prof.-dr.-olaf-zimmermann}{%
\subsubsection*{Prof. Dr. Olaf
Zimmermann}\label{prof.-dr.-olaf-zimmermann}}

\hypertarget{elias-ux5cux26-oli}{%
\subsubsection*{Elias \& Oli}\label{elias-ux5cux26-oli}}