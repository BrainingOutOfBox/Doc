\section{Testprotokoll}
\subsection{Zweck dieses Dokuments}
Um sicher zu stellen, dass unsere Cross-Plattform Applikation in ihrer Kernfunktionalität richtig funktioniert, wurde ein systematischer Test anhand des nachfolgenden Testprotokolls durchgeführt.

\subsection{Verweise}
Die Testfälle leiten sich von den Fully-Dressed Use Cases aus Kapitel \ref{par:fully_dressed_use_cases} ab.

\subsection{Testfälle}

\renewcommand{\arraystretch}{1.35}
\begin{center}
	\begin{longtable}{| p{1cm} | p{5cm} | p{5cm} | p{2cm} |}
		\hline
		\multicolumn{4}{|c|}{\textbf{Use Case 7: View Brainstorming Finding}}\\
		\hline\hline
		Test Nr & Beschreibung & Erwartetes Resultat & Bestanden (J/N) \\
		\hline
		1 & Als Participant möchte ich als Letzter die letzte Runde abschliessen. & Das System speichert meine Notizen auf dem Server und zeigt mir eine Meldung an, dass das Finding einsehbar ist. & INSERT HERE \\
		\hline
		2 & Als Participant möchte ich die Resultate der Gruppe ansehen. & Das System zeigt mir eine Übersicht mit allen Notizen der Teilnehmer. & INSERT HERE\\
		\hline
		3 & Als Participant möchte ich nicht als Letzter die letzte Runde abschliessen. & Das System speichert meine Notizen auf dem Server und zeigt mir die verbleibende Zeit an. & INSERT HERE\\
		\hline
		4 & Nach Ablauf der Zeit meldet mir das System, dass das Finding einsehbar ist. & Das System zeigt mir eine Übersicht mit allen Notizen der Teilnehmer. & INSERT HERE \\
		\hline
		5 & Als Participant möchte ich nicht direkt zu den Resultaten der Gruppe navigieren sondern zurück zum Homescreen gelangen. & Das System zeigt mir den Homescreen an. Von dort aus gelange ich aber auch zu den Notizen der Teilnehmer. & INSERT HERE  \\
		\hline
	\end{longtable}
\end{center}


\renewcommand{\arraystretch}{1.35}
\begin{center}
	\begin{longtable}{| p{1cm} | p{5cm} | p{5cm} | p{2cm} |}
		\hline
		\multicolumn{4}{|c|}{\textbf{Use Case 8: Create Brainwave}}\\
		\hline\hline
		Test Nr & Beschreibung & Erwartetes Resultat & Bestanden (J/N) \\
		\hline
		1 & Als Participant möchte ich Notizen während der laufenden Rundenzeit zum aktuellen Problem erfassen. & Das System zeigt meine Notizen an. & INSERT HERE \\
		\hline
		2 & Als Participant möchte ich meine Notizen frühzeitig abgeben. & Das System persistiert meine Notizen und zeigt mir eine Bestätigung an. & INSERT HERE\\
		\hline
		3 & Als Participant möchte ich meine Notizen nicht frühzeitig abgeben. & Das System zeigt mir eine Meldung an, dass die Zeit der aktuellen Runde abgelaufen ist. Es persistiert die bis zu dem Zeitpunkt erfassten Notizen. & INSERT HERE\\
		\hline
	\end{longtable}
\end{center}

\renewcommand{\arraystretch}{1.35}
\begin{center}
	\begin{longtable}{| p{1cm} | p{5cm} | p{5cm} | p{2cm} |}
		\hline
		\multicolumn{4}{|c|}{\textbf{Use Case 9: Start Brainstorming}}\\
		\hline\hline
		Test Nr & Beschreibung & Erwartetes Resultat & Bestanden (J/N) \\
		\hline
		1 & Als Moderator möchte ich ein Brainstorming-Session starten. & Das System überprüft zunächst, ob alle Einstellungen stimmen und startet dann die Brainstorming-Session. & INSERT HERE \\
		\hline
		2 & Als Participant möchte ich ein Brainstorming-Session starten. & Als Participant ist es nicht möglich eine Brainstorming-Session zu starten & INSERT HERE\\
		\hline
	\end{longtable}
\end{center}
