\thispagestyle{empty}
\section*{Abstract}
Mobile Applikationen für das Smartphone haben in den letzten Jahren immer mehr an Bedeutung gewonnen. Das liegt unter anderem daran, dass sie einfach zu bedienen sind und man sein eigenes Smartphone immer und überall dabei hat. Dieser Umstand hat uns in der vorliegenden Arbeit dazu bewogen, die Methode 635 als Cross-Plattform Applikation für iOS und Android zu implementieren. Die Methode 635 ist eine Kreativitäts- und Brainwriting-Technik, welche die Entwicklung von neuen, ungewöhnlichen Ideen für Problemlösungen in der Gruppe fördert. Die Methode wird im Modul \grqq Projekt und Qualitätsmanagement\grqq{} an der HSR vorgestellt.

Dafür haben wir eine Vorstudie angefertigt, in der wir uns zunächst genauer mit der Methode 635 auseinander gesetzt haben. Ebenfalls beschäftigten wir uns während der Vorstudie mit der Frage, ob sich Xamarin für unser Projekt überhaupt anbietet und welches der beiden Umsetzungsarten (Xamarin.Forms oder Xamarin native) für unser Projekt besser geeignet ist.

Aus diesen und weiteren Analysen und Konzepten entwickelten wir einen einfachen Prototypen. Zweck des Prototypen war es primär zu klären, ob unsere theoretischen Überlegungen auch praktisch funktionierten und dieser in der Lage war, die definierten NFAs abzudecken. Da dies der Fall war, entwickelten wir basierend auf unserem Prototypen das gesamte System weiter.

Aus der Arbeit ist eine lauffähige Cross-Plattform Applikation für iOS und Android hervorgegangen, welche es Benutzern ermöglicht, die Methode 635 auf ihre Probleme anzuwenden. Weiter dokumentierten wir Erkenntnisse über Herausforderungen, die während der Umsetzung aufgetreten sind. So können unsere Erfahrungen anderen Entwicklern bei ähnlichen Projekten helfen oder einzelne Risiken gar komplett verringern. 