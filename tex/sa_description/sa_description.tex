\section{Aufgabenstellung}

\subsection{Ausgangslage}
Gerade in der IT-Welt ist das Finden von Lösungen für ein neues aber auch für ein bestehendes Problem täglicher Bestandteil des Jobs. Durch den Einsatz von verschiedenen Lösungs\-findungs\-methoden kann der Ideenfindung oftmals auf die Sprünge geholfen werden. Es gibt dazu die verschiedensten Ansätze und Methoden, in dieser Studienarbeit soll allerdings die Methode 635 als Ausgangslage dienen. Die Motivation besteht darin, eine Cross-Platform App zu programmieren, welche die Methode 635 als mobile App für Android und iOS umsetzt. Dabei sollen moderne Technologien zum Einsatz kommen, welche es den Anwendern ermöglichen schneller und einfacher eine Lösung für ein Problem zu erarbeiten. 

\subsection{Ziele der Arbeit und Liefergegenstände}

Es wird erwartet, dass bis zum Ende des Projektes eine Cross-Plattform Applikation mit Xamarin umgesetzt wird, welche es den Benutzern ermöglicht, die Methode 635 auf deren Probleme anzuwenden. Damit die App einen Mehrwert gegenüber der Papierversion bietet, soll es z.B. möglich sein, die Anzahl der Teilnehmer variabel zu bestimmen oder verschiedene Medien (Text, Video, Bilder, etc.) zu verwenden bzw. einzubinden. Ausserdem ist das Persistieren der bearbeiteten Problemstellungen aus Sicht des Kunden viel einfacher als mit Papier.
Da es sich um eine mobile Anwendung handelt, ist es für die Anwender zudem ein Leichtes die Methode 635 zu nutzen auch wenn sie nich am selben Ort sind oder die Lösungsvorschläge nicht zur selben Zeit bearbeiten. 
Das übergeordnete Ziel soll also darin bestehen, die Papierversion für diese Methodik zu schlagen. Dabei spielen Erfolgsfaktoren wie einfache und intuitive Bedienung der App und ein unkompliziertes Reporting eine wichtige Rolle.
