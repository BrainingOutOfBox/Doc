\section{Management Summary}

\subsection{Ausgangslage}
In einer Zeit, in der der technische Fortschritt immer mehr an Bedeutung gewinnt, hat man das Gefühl alles könne automatisiert und durch die Technik schneller und präziser erledigt werden. Doch bei der Aufgabe eine Lösung für ein bestimmtes Problem zu finden, ist der menschliche Ideenreichtum unumgänglich. Durch den Einsatz von verschiedensten Innovationsmethoden kann der Ideenfindung häufig etwas nachgeholfen werden. Dazu zählt auch die Methode 635. Die Methode 635 ist eine Kreativitäts- und Brainwriting-Technik, welche die Entwicklung von neuen, ungewöhnlichen Ideen für Problemlösungen in der Gruppe fördert.

Nach heutigem Wissenstand hat sich dafür noch keine mobile Applikation in den App Stores von Google oder Apple durchgesetzt. Daher ist es das Ziel dieser Arbeit, die Methode 635 als mobile Cross-Plattform Applikation zu konzipieren und zu implementieren. Die Möglichkeiten der Technik sollen dazu genutzt werden, eine Lösung für ein Problem schneller und einfacher zu erarbeiten.

\subsection{Vorgehen}
Das Vorgehen für unser Projekt lässt sich in zwei Phasen einteilen. In der ersten Phase konzentrierten wir uns primär auf Recherchearbeiten und konzeptionelle Arbeiten. Dabei führten wir eine Vorstudie durch, in der wir grundsätzliche Analysen durchführten und Fragen klärten. Des Weiteren konzipierten wir zunächst theoretisch wie wir die Methode 635 als Cross-Plattform Applikation umsetzen sollten. Eine weitere Aufgabe war das Verfassen der funktionalen sowie nicht-funktionalen Anforderungen. Um unsere Vorüberlegungen abzusichern, programmierten wir zu Ende dieser Phase einen einfach Prototypen.

In der zweiten Phase ging es dann darum, alle unsere Überlegungen und Designs komplett umzusetzen. Da unser Prototyp wie gewünscht funktionierte, entwickelten wir basierend darauf alle weiteren Funktionen. Damit wir sicherstellen konnten, dass die Applikation wie gewünscht funktionierte, testeten wir während der gesamten Dauer manuell unser System.

\subsection{Ergebnisse}
Aus der Arbeit ist eine lauffähige Cross-Plattform Applikation für iOS und Android hervorgegangen, welche es Benutzern ermöglicht, die Methode 635 auf ihre Probleme anzuwenden. Die Applikation ist sehr variabel gestaltet. So kann die Anzahl an Teilnehmer pro Gruppe sowie die Anzahl an Ideen pro Runde und die Startdauer pro Runde individuell bestimmt werden. 

Durch das Persistieren in einer zentralen Datenbank ist es dem Endnutzer zudem möglich, seine erarbeiteten Ideen jederzeit abzurufen und weiter zu verwenden. 

Die Möglichkeit andere Medien, wie Video, Bilder oder Zeichnungen zu verwenden, werden in dieser Version allerdings noch nicht unterstützt.
